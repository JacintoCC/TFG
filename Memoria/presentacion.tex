\documentclass[leqno]{beamer}
\usepackage[spanish,activeacute]{babel}
\usepackage[utf8]{inputenc}
\usepackage{amsfonts}
\usepackage{enumerate}
\usepackage{amsthm}
\usepackage{graphicx}
\usepackage{listings}

\usetheme{CambridgeUS}
\usecolortheme{spruce}
\setbeameroption{show notes} %un-comment to see the notes

\title{Trabajo de Fin de Grado}
\subtitle{Evaluación del estado del arte en técnicas estadísticas para el análisis comparativo de algoritmos de aprendizaje automático}
\author{Jacinto Carrasco Castillo}

\newtheoremstyle{definition_wo_parentheses}
  {\topsep}% measure of space to leave above the theorem. E.g.: 3pt
  {\topsep}% measure of space to leave below the theorem. E.g.: 3pt
  {}% name of font to use in the body of the theorem
  {0pt}% measure of space to indent
  {\bfseries}% name of head font
  {.}% punctuation between head and body
  { }% space after theorem head; " " = normal interword space
  {\thmname{#1}\thmnumber{ #2.}\thmnote{ #3}}
  
  
\theoremstyle{definition_wo_parentheses}
\newtheorem{definicion}{Definición}


\AtBeginSection[]{
  \begin{frame}
  \vfill
  \centering
  \begin{beamercolorbox}[sep=8pt,center,shadow=true,rounded=true]{title}
    \usebeamerfont{title}\insertsectionhead\par%
  \end{beamercolorbox}
  \vfill
  \end{frame}
}

\begin{document}

\begin{frame}
\titlepage
\end{frame}

\begin{frame}
\tableofcontents
\end{frame}

\section{Planteamiento}


\begin{frame}{Descripción del problema}
	La comparación de algoritmos debe realizarse de manera 
rigurosa para reducir la aleatoriedad asociada a los 
experimentos. Para realizar una correcta comparación es 
necesario:
	
\begin{itemize}
	\item Selección de medida del rendimiento.
	\item Validación cruzada y remuestreo. 
	\item Test estadísticos.
\end{itemize}
\end{frame}

\begin{frame}{Objetivos}
\begin{itemize}
	\item Estudio de los test estadísticos disponibles.
	\item Integración de herramientas informáticas existentes
	 para la aplicación de estos test.
	\item Comparación de las propiedades de los test.
\end{itemize}
\end{frame}

\section{Contenido matemático}

\subsection{Test paramétricos}
\begin{frame}{Test paramétricos}

	Suponen que la muestra pertenece a una distribución conocida. 

\begin{description}
\item[Test binomial para una muestra] Comprobación de que 
	el rendimiento para un problema sea igual a un valor 
	$\theta_0$.
\item[$t$-test para muestras apareadas]  Comparación de la 
	media de dos muestras. Comparación de dos algoritmos
	para un conjunto de datos.
\item[ANOVA] Comparación de la media de múltiples algoritmos
	en múltiples problemas. Trata la varianza
	en un grupo, entre grupos y la combinación de ellas.
\end{description}
\end{frame}

\subsection{Test no paramétricos}

\begin{frame}{Test no paramétricos}{Comparación con test paramétricos}
\begin{itemize}
\item Los test no paramétricos no suponen la pertenencia 
	de la distribución a ninguna familia de distribuciones.
\item Las hipótesis para aplicar estos test son más
	generales. 
\item Si se dan las hipótesis necesarias para los test
	paramétricos, tienen una menor potencia.
\item Si se disponen de pocos datos las hipótesis de los
	test paramétricos no suelen darse y los test no
	paramétricos suplen la falta de potencia con mayor
	precisión.
\end{itemize}		
\end{frame}

\begin{frame}{Test no paramétricos}
	Tipos de test no paramétricos:
\begin{itemize}
\item Test de aleatoriedad: basados en el número de rachas.
\item Test de bondad del ajuste: Test chi cuadrado, 
	Kolmogorov-Smirnov.
\item Análisis del conteo de datos: Test de McNemar.
\item Test basado en una muestra y muestras apareadas: Test
	de signo, test de rangos con signos de Wilcoxon. 
\item Análisis bidimensional de la varianza: Test de 
	Friedman, modificación de Iman-Davenport, test de 
	rangos alineados de Friedman, test de Quade.
\end{itemize}
\end{frame}

\begin{frame}{Test no paramétricos}{Procedimientos \textit{post-hoc}}
	Los test que comparan múltiples algoritmos indican la
existencia de diferencias entre ellos. Necesitan un
test adicional para indicar dónde están estas diferencias.

\begin{itemize}
\item $p$-valores ajustados: Al realizar múltiples comparaciones aumenta la probabilidad de cometer un error de tipo I. Hay distintos métodos para ajustar los $p$-valores obtenidos.
\end{itemize}
\end{frame}

\begin{frame}{Test no paramétricos}{Test basados en permutaciones}
	Son test no paramétricos. La hipótesis nula es que una muestra $\mathbf{x}$ proviene de una misma población.\\
	Suponiendo $H_0$ cierta, los individuos de cada 
subconjunto de la muestra se puede intercambiar por los de 
otro subconjunto.
\begin{definicion}[Principio de los test basados en permutaciones] Si dos experimentos toman valores en el mismo
espacio muestral $\Omega$ con distribuciones $P_1, P_2$ dan
el mismo conjunto de datos, las inferencias condicionales
a los datos usando el mismo estadístico deben ser la misma.
\end{definicion}
\end{frame}


\subsection{Test bayesianos}
\begin{frame}{Comparación con THN}
	La inferencia bayesiana ajusta un modelo de probabilidad
a los datos y obtiene una distribución sobre los parámetros
del modelo. Las diferencias con los test de hipótesis nula son:

\begin{itemize}
\item Se evitan decisiones dicotómicas marcadas por $\alpha$.
\item Los THN no estiman la probabilidad de la hipótesis.
\item Con suficientes datos se rechaza casi toda hip. nula.
\item No se tiene en cuenta magnitud de la diferencia ni incertidumbre.
\item No se obtiene información si no se rechaza la hipótesis nula.
\end{itemize}

\end{frame}

\begin{frame}{Test bayesianos}
\begin{description}
\item[$t$-test bayesiano correlado] Comparación de dos
dos algoritmos en un único conjunto de datos. Tiene en cuenta
la correlación de los conjuntos de entrenamientos en CV.
Se obtiene una distribución $T$ de Student sobre la
diferencia de las medias.
\item[Test bayesiano de signo] Versión bayesiana del test de 
signo. Se obtiene distribución sobre la probabilidad de que 
la diferencia entre algoritmos sea menor que 0, prácticamente 
nula o mayor que 0.
\item[Test bayesiano de rangos alineados] Versión bayesiana 
del test de rangos alineados. No se obtiene una distribución clara de los parámetros, pero se puede obtener una muestra.
\end{description}
\end{frame}

\section{Contenido informático}

\begin{frame}{rNPBST}
	Paquete desarrollado en \texttt{R}. Incluye:
	
\begin{itemize}
\item Conjunto de datos sobre la aplicación de 5 algoritmos 
	en 29 conjuntos de datos para realizar los test.
\item Test no paramétricos de la biblioteca \texttt{JavaNPST} 
	a los que se accede mediante \texttt{rJava}.
\item Test bayesianos y representación gráfica de 
	sus resultados.
\end{itemize}
\end{frame}

\begin{frame}[fragile]{rNPBST}{Instalación}
Paquete de \texttt{R} disponible en 
\begin{center}
\url{https://github.com/JacintoCC/TFG/tree/master/rNPBST}
\end{center}
Para la instalación, ejecutar donde se encuentre la carpeta con el software:
	\begin{verbatim}
	> R CMD build rNPBST
	> R CMD INSTALL rNPBST
	\end{verbatim}
\end{frame}

\begin{frame}[fragile]{Aplicación de test}
\begin{verbatim}
 > data("results")
 > ft <- friedman.test(results)
 > ft$htest
 	Friedman test
 data:  
s = 2812.000, q = 39.056, p-value = 6.789e-08
\end{verbatim}
Con estos resultados rechazaríamos la hipótesis nula de la 
equivalencia de los algoritmos. El test nos devuelve en el 
parámetro \texttt{report} también la suma del ránquin
medio de cada algoritmo. 
\end{frame}


\begin{frame}[fragile]{Aplicación de test}{Test bayesianos - $t$-Test bayesiano correlado}
   \begin{verbatim}
   > dataset.index <- 13
   > correlatedBayesianT.test(results.lr[dataset.index, ],
                              results.rf[dataset.index, ])
   \end{verbatim}
   \begin{center}
   \includegraphics[width=0.4\textwidth]{imagenes/LR-RF-hayes-roth.pdf}
   \end{center}
\end{frame}

\begin{frame}[fragile]{Aplicación de test}{Test bayesianos - Test bayesiano de rangos con signo}
   \begin{verbatim}
   > bayesianSignedRank.test(results$KNN,
                            results$neural.network)
   \end{verbatim}
   \begin{center}
   \includegraphics[width=0.5\textwidth]{imagenes/BSR-KNN-NNET.pdf}
   \end{center}
\end{frame}

\end{document}
