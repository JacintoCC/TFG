\documentclass[a4paper,11pt]{book}
%\documentclass[a4paper,twoside,11pt,titlepage]{book}
\usepackage{listings}
\usepackage[utf8]{inputenc}
\usepackage[spanish]{babel}

% \usepackage[style=list, number=none]{glossary} %
%\usepackage{titlesec}
%\usepackage{pailatino}

\decimalpoint
\usepackage{dcolumn}
\newcolumntype{.}{D{.}{\esperiod}{-1}}
\makeatletter
\addto\shorthandsspanish{\let\esperiod\es@period@code}
\makeatother


%\usepackage[chapter]{algorithm}
\RequirePackage{verbatim}
%\RequirePackage[Glenn]{fncychap}
\usepackage{fancyhdr}
\usepackage{graphicx}
\usepackage{afterpage}

\usepackage{longtable}

\usepackage[colorlinks=true,linkcolor=black]{hyperref} %referencia

\usepackage{float}

% ********************************************************************
% Re-usable information
% ********************************************************************

\newcommand{\myTitle}{Evaluación del estado del arte en técnicas estadísticas para el análisis comparativo de algoritmos de aprendizaje automático}
\newcommand{\myDegree}{Grado en Ingeniería Informática y Matemáticas}
\newcommand{\myName}{Jacinto Carrasco Castillo}
\newcommand{\myProf}{Francisco Herrera Triguero}
\newcommand{\myOtherProf}{María del Mar Rueda García}
%\newcommand{\mySupervisor}{Put name here\xspace}
\newcommand{\myFaculty}{Escuela Técnica Superior de Ingenierías Informática y de Telecomunicación, Falcultad de Ciencias}
\newcommand{\myFacultyShort}{}
\newcommand{\myDepartment}{}
\newcommand{\myUni}{\protect{Universidad de Granada}}
\newcommand{\myLocation}{Granada}
\newcommand{\myTime}{Septiembre, 2016}
\newcommand{\myVersion}{Version 0.1}


\hypersetup{
pdfauthor = {\myName (jacintocc (en) (punto) ugr (punto) es)},
pdftitle = {\myTitle},
pdfsubject = {},
pdfkeywords = {palabra_clave1, palabra_clave2, palabra_clave3, ...},
pdfcreator = {LaTeX con el paquete ....},
pdfproducer = {pdflatex}
}

% ********************************************************************
%\hyphenation{}


%\usepackage{doxygen/doxygen}
%\usepackage{pdfpages}
\usepackage{url}
\usepackage{colortbl,longtable}
\usepackage[stable]{footmisc}
\usepackage{amsthm}
%\usepackage{index}

%\makeindex
%\usepackage[style=long, cols=2,border=plain,toc=true,number=none]{glossary}
% \makeglossary

% Definición de comandos que me son útiles:
%\renewcommand{\indexname}{Índice alfabético}
%\renewcommand{\glossaryname}{Glosario}

\pagestyle{fancy}
\fancyhf{}
\fancyhead[LO]{\leftmark}
\fancyhead[RE]{\rightmark}
\fancyhead[RO,LE]{\textbf{\thepage}}
\renewcommand{\chaptermark}[1]{\markboth{\textbf{#1}}{}}
\renewcommand{\sectionmark}[1]{\markright{\textbf{\thesection. #1}}}

\setlength{\headheight}{1.5\headheight}

\newcommand{\HRule}{\rule{\linewidth}{0.5mm}}
%Definimos los tipos teorema, ejemplo y definición podremos usar estos tipos
%simplemente poniendo \begin{teorema} \end{teorema} ...
\newtheorem{ejemplo}{Ejemplo}[chapter]
\theoremstyle{plain}
\newtheorem{teorema}{Teorema}[chapter]
\newtheorem{lema}[teorema]{Lema}
\newtheorem{proposicion}[teorema]{Proposición}
\newtheorem{corolario}[teorema]{Corolario}

% Definición de estilo de definición



\newtheoremstyle{definition_wo_parentheses}
  {\topsep}% measure of space to leave above the theorem. E.g.: 3pt
  {\topsep}% measure of space to leave below the theorem. E.g.: 3pt
  {}% name of font to use in the body of the theorem
  {0pt}% measure of space to indent
  {\bfseries}% name of head font
  {.}% punctuation between head and body
  { }% space after theorem head; " " = normal interword space
  {\thmname{#1}\thmnumber{ #2.}\thmnote{ #3}}
  
  
\theoremstyle{definition_wo_parentheses}	
\newtheorem{definicion}{Definición}[chapter]

\definecolor{gray97}{gray}{.97}
\definecolor{gray75}{gray}{.75}
\definecolor{gray45}{gray}{.45}
\definecolor{gray30}{gray}{.94}

\lstset{ frame=Ltb,
     framerule=0.5pt,
     aboveskip=0.5cm,
     framextopmargin=3pt,
     framexbottommargin=3pt,
     framexleftmargin=0.1cm,
     framesep=0pt,
     rulesep=.4pt,
     backgroundcolor=\color{gray97},
     rulesepcolor=\color{black},
     %
     stringstyle=\ttfamily,
     showstringspaces = false,
     basicstyle=\scriptsize\ttfamily,
     commentstyle=\color{gray45},
     keywordstyle=\bfseries,
     %
     numbers=left,
     numbersep=6pt,
     numberstyle=\tiny,
     numberfirstline = false,
     breaklines=true,
   }
 
% minimizar fragmentado de listados
\lstnewenvironment{listing}[1][]
   {\lstset{#1}\pagebreak[0]}{\pagebreak[0]}

\lstdefinestyle{CodigoC}
   {
	basicstyle=\scriptsize,
	frame=single,
	language=C,
	numbers=left
   }
\lstdefinestyle{CodigoC++}
   {
	basicstyle=\small,
	frame=single,
	backgroundcolor=\color{gray30},
	language=C++,
	numbers=left
   }
\lstdefinestyle{CodigoR}
   {
	basicstyle=\small,
	frame=single,
	backgroundcolor=\color{gray30},
	language=R,
	numbers=left
   }
 
\lstdefinestyle{Consola}
   {basicstyle=\scriptsize\bf\ttfamily,
    backgroundcolor=\color{gray30},
    frame=single,
    numbers=none
   }


\newcommand{\bigrule}{\titlerule[0.5mm]}


%Para conseguir que en las páginas en blanco no ponga cabecerass
\makeatletter
\def\clearpage{%
  \ifvmode
    \ifnum \@dbltopnum =\m@ne
      \ifdim \pagetotal <\topskip
        \hbox{}
      \fi
    \fi
  \fi
  \newpage
  \thispagestyle{empty}
  \write\m@ne{}
  \vbox{}
  \penalty -\@Mi
}
\makeatother

\usepackage{pdfpages}
\usepackage{gensymb}
\usepackage{amsfonts}
\usepackage{amsmath}




% ****************************************************************
% DOCUMENTO
% ****************************************************************

\begin{document}
\pagenumbering{roman}
\input{portada/portada}
\chapter*{}
%\thispagestyle{empty}
%\cleardoublepage

%\thispagestyle{empty}

\input{portada/portada_2}



\cleardoublepage
\thispagestyle{empty}

\begin{center}
{\large\bfseries \myTitle }\\
\end{center}
\begin{center}	
	\myName \\
\end{center}

%\vspace{0.7cm}
\noindent{\textbf{Palabras clave}: aprendizaje automático,
	evaluación de algoritmos, test paramétricos, test no 
	paramétricos, test basados en permutaciones, test 
	bayesianos, R}\\

\vspace{0.7cm}
\noindent{\textbf{Resumen}}\\

	En este trabajo se pretende realizar un estudio sobre
	la aplicación de test estadísticos en la comparación
	del rendimiento de algoritmos de aprendizaje automático.
	Se incluye la descripción matemática de los test más
	utilizados tanto a la hora de detectar diferencias 
	entre los algoritmos como para evaluar el cumplimiento
	de ciertas características de los datos para la
	realización de los test. Se presentan test de 
	hipótesis nula basados en la estadística frecuentista,
	tanto paramétricos como no paramétricos, y test
	bayesianos. Para mostrar un ejemplo del problema, se 
	muestra la aplicación de test estadísticos sobre
	los resultados de varios algoritmos de clasificación
	sobre diversos conjuntos de datos y se buscan las 
	diferencias en el rendimiento de estos algoritmos.
	Este estudio nos permitirá observar las propiedades
	de cada test y realizar la comparación entre la
	potencia de cada uno de ellos.
\cleardoublepage


\thispagestyle{empty}


\begin{center}
{\large\bfseries Evaluation of state of the art stadistic techniques for machine learning algorithms' comparative analysis}\\
\end{center}
\begin{center}
	\myName	\\
\end{center}

%\vspace{0.7cm}
\noindent{\textbf{Keywords}: Machine learning, algorithms' evaluation, 
	parametric tests, non parametric tests, permutation tests, 
	bayesian tests, R}\\

\vspace{0.7cm}
\noindent{\textbf{Abstract}}\\

	In recent years the interest in machine learning has 
meant an increasing amount in the proposed machine learning 
algorithms. Nevertheless, these new algorithms need to be 
tested and the properties and results must be proved, so in 
order to achieve this objective statistical tests are the 
best approach.\\
	In this project, we present a set of tests commonly used 
in this problem, with a practical case of study analysing the 
power of each test. \\
	In the first part of this memory, we present the 
mathematical base of the statistical tests. This part 
contains an introduction about measuring the algorithms’ 
performance and the ways to obtein a reliable measure of this 
performance. The tests presented are classified in three 
categories: parametric, non-parametric and Bayesian tests. In 
each categorie we include tests with different goals: 
comparison between two algorithms in one data set, two 
algorithms in multiple data sets or between multimple 
algoritms in multiple data sets. The first set of statistical 
tests presented are the parametric ones, which are widely 
known and used, like t-test or ANOVA, but presents some 
faults related with the necessary conditions to apply the 
parametric tests, which are rarely given. \\
	We describe these faults and present the non parametric 
tests, which do not need these strict conditions about the 
population like normality or homocedasticity but relaxed 
conditions like the symmetry of the population. Non 
parametric test are usually less powerful than parametric one 
due to in non parametric test we have less information about 
the underlying population of the data, but this lack of power 
is compensated with the lack of precision in the parametric 
tests when the number of data is low or the conditions are 
not satisfied. The test presented in the non parametric 
chapter allows us to obtain some valuable information as the 
randomness of a sample or the goodness of fit of the data to 
a given population, which are conditions to use the previous 
parametric tests. Moreover, non parametric tests as Wilcoxon 
Signed-ranks for comparing two algorithms or Friedman test 
for multiple algorithms over multiple data sets are presented 
in this chapter with some improvements like Iman-Davenport 
test or Quade test, which takes account of the different 
difficulties of the data sets. Once the null hypothesis of an 
equal performance of all algorithms in the study is rejected, 
we need a post-hoc method to find where is the difference 
between algorithms, so we introduce some procedures, like 
Finner’s, Hommel’s or Li’s procedures. As a special case of 
non parametric tests, we include a general description of the 
base of permutation tests, which start with a sample of two 
populations and collect information against the null 
hypothesis making permutations of the sample.\\ 
	As a complete different approach of the comparison, we 
present Bayesian tests. Null hypothesis tests present some 
problems mainly related with the understanding of what 
information they give. More frequently that it should 
happens, researchers thinks that the p-value given by a null 
hypothesis test corresponds to the probability, given a 
sample, that the null hypothesis is true instead of the 
actual meaning, that is the probability, assuming the null 
hypothesis is true, of obtaining a sample as extreme or more 
than our sample. The wanted statement is not achieved with 
frequentist statistical methods but Bayesian ones, so we 
present a series of differences between null hypothesis tests 
and Bayesians tests. Moreover, we present a Bayesian version 
of frequentist tests as t-test, sign test and Wilcoxon 
signed-rank test.\\
	We dedicate the second part of the memory to the 
experimental study of the comparison between algorithms. The 
first chapter contains a revision of state of the art 
statistical techniques applied in recent years as well as 
most influential articles in the field. 



\chapter*{}
\thispagestyle{empty}

\noindent\rule[-1ex]{\textwidth}{2pt}\\[4.5ex]

Yo, \textbf{\myName}, alumno de la titulación \myDegree de la \textbf{\myFaculty de la \myUni}, con DNI 32056356, autorizo la
ubicación de la siguiente copia de mi Trabajo Fin de Grado en la biblioteca de ambos centros para que pueda ser
consultada por las personas que lo deseen.

\vspace{6cm}

\noindent Fdo: \myName

\vspace{2cm}

\begin{flushright}
Granada a 9 de septiembre de 2016.
\end{flushright}


\chapter*{}
\thispagestyle{empty}

\noindent\rule[-1ex]{\textwidth}{2pt}\\[4.5ex]

D. \textbf{ \myProf }, Profesor del Departamento de Ciencias de la Computación e Inteligencia Artificial de la Universidad de Granada.

\vspace{0.5cm}

D. \textbf{ \myOtherProf}, Profesora del Departamento de Estadística e Investigación Operativa de la Universidad de Granada.


\vspace{0.5cm}

\textbf{Informan:}

\vspace{0.5cm}

Que el presente trabajo, titulado \textit{\textbf{\myTitle}},
ha sido realizado bajo su supervisión por \textbf{\myName}, y autorizamos la defensa de dicho trabajo ante el tribunal
que corresponda.

\vspace{0.5cm}

Y para que conste, expiden y firman el presente informe en Granada a X de septiembre de 2016.

\vspace{1cm}

\textbf{Los directores:}

\vspace{5cm}

\noindent \textbf{\myProf \ \ \ \ \ \myOtherProf}

\chapter*{Agradecimientos}
\thispagestyle{empty}

       \vspace{1cm}


Poner aquí agradecimientos...


%\frontmatter
\tableofcontents
%\listoffigures
%\listoftables
%
%\mainmatter
%\setlength{\parskip}{5pt}


\chapter{Introducción}
\pagenumbering{arabic}
\setcounter{page}{1}

\section{Contextualización}
	
	El interés creciente en el campo del aprendizaje
automático ha significado un aumento en la propuesta
de nuevos algoritmos que pretenden mejorar a los 
ya existentes. Se describe a continuación el proceso
general para poder comparar algoritmos, basándonos en
las recomendaciones realizadas por Alpaydin 
(\cite{Alpaydin:2010:IML:1734076}) y Japkowicz  
(\cite{DBLP:books/cu/Japkowicz2011}).\\
	A la hora de evaluar un algoritmo,
nos interesa conocer el error asociado a un problema y 
su relación con respecto a otros. Sin embargo, debemos ser 
conscientes de que para ello debemos diseñar y realizar un 
experimento cuyos resultados sean fiables y minimizar el 
efecto producido por las circunstancias en las que se 
realizan o los datos disponibles. Por ejemplo, el error 
obtenido en los datos utilizados para entrenar el modelo 
(error de entrenamiento) será menor que el error en unos 
datos que no han participado en este entrenamiento y son 
usados para evaluar el modelo (error de test). Cabe mencionar 
también que el error de entrenamiento no es comparable debido 
a que modelos más complejos se ajustarán más a los datos de 
entrenamiento, sin que eso signifique que el modelo se 
comporte mejor de manera global. Hay algoritmos que se 
comportan de manera no determinista, y por tanto deseemos 
realizar varias ejecuciones para estimar circunstancias 
aleatorias. En definitiva, debemos basar nuestra evaluación 
del algoritmo en la distribución (desconocida) del error. 
Para una correcta estimación del error,  es importante tener en 
cuenta lo siguiente:
	
	\begin{itemize}
	\item El error depende del problema. El teorema de 
		\textit{No Free Lunch} afirma que dados dos 
		algoritmos su rendimiento medio para todos los 
		posibles problemas es el mismo, lo que se traduce en 
		que según el problema se puede dar que un algoritmo 
		sea mejor que otro o al contrario.
	\item La partición de la base de datos referente al 
		problema en datos de entrenamiento y de test se 
		realiza para obtener una idea más acertada del error 
		cometido por el algoritmo para el problema. Sin 
		embargo, una vez que se considere válido el algoritmo 
		se entrenará con todos los datos disponibles.
	\item Para validar el algoritmo, se deben utilizar datos 
		que no hayan sido utilizados anteriormente de ninguna 
		manera.
	\item Normalmente, para la comparación del rendimiento de 
		algoritmos se usa el error cometido en la 
		clasificación, aunque se podrían considerar otros 
		factores como el tiempo necesario para entrenar el 
		algoritmo, su simplicidad...
	\end{itemize}
	
\subsubsection{Diseño de experimentos}

	Al hablar de experimentos nos referimos al test o a la 
serie de test donde se manejan factores que afectan a la 
salida. Pretendemos minimizar aquellos factores que no se 
controlan y obtener resultados estadísticamente 
significativos. Para ajustar parámetros propios del modelo de 
aprendizaje automático que estemos utilizando existen 
diferentes aproximaciones: la \textit{mejor suposición} 
(donde quien realiza el experimento ajusta los parámetros 
basándose en su conocimiento sobre el problema), \textit{un 
factor cada vez} (se supone que los factores son 
independientes y se prueban diferentes valores para cada 
parámetro de uno en uno, mientras los demás se mantienen en 
una posición base) y \textit{diseño factorial} (se crea una 
rejilla con diferentes posibles valores para cada parámetro y 
se prueba en ellos, es más costoso).\\

	Para minimizar el número de ejecuciones necesarias, se 
suele utilizar conocimiento previo para reducir las 
combinaciones a probar. Otra estrategia consiste en el diseño 
de la superficie de respuesta. Se considera la siguiente 
situación:
		\[ r = g( f_1, \dots, f_F | \phi ), \]
	donde $r$ es la respuesta, $g$ el modelo utilizado, 
$f_1, \dots f_F$ los factores y $\phi$ la estimación empírica 
del modelo para cada configuración particular probada. El 
procedimiento consiste en ir añadiendo a $\phi$ las 
evaluaciones realizadas, ajustar $g$ según $\phi$ y buscar su 
máximo (o mínimo), evaluar el algoritmo en ese punto y 
ajustar $\phi$.\\
	En la realización de experimentos es necesaria la 
aleatorización a la hora de probar las diferentes 
configuraciones de parámetros (no tanto en nuestro contexto, 
pero el orden de las configuraciones al realizar experimentos 
donde interviene maquinaria por ejemplo podría interferir en los 
resultados debido a factores como la temperatura de la 
máquina). Otro factor a tener en cuenta es que es necesaria 
la repetición de las ejecuciones y promediar los resultados 
para así reducir el impacto de factores aleatorios. Para 
reducir la variabilidad de factores que no dependen del 
propio algoritmo a evaluar, como las bases de datos o las 
particiones realizadas, en la comparación de varios 
algoritmos utilizamos las mismas particiones para cada uno.

\paragraph{Directrices a la hora de realizar un experimento}
	\begin{enumerate}
	\item Fijar la intención del estudio
	\item Seleccionar las variables de respuesta
	\item Seleccionar los factores con los que se realizará 
		el experimento y los niveles a comprobar
	\item Diseño del experimento: La partición para 
		entrenamiento y test depende del tamaño de la base de 
		datos. Si la base de datos es pequeña puede haber una 
		alta variabilidad y obtener resultados no 
		concluyentes.
	\item Efectuar el experimento: Antes de llevar a cabo un 
		experimento de gran magnitud es aconsejable probar 
		pequeños conjuntos de datos para probar que funciona. 
		También es interesante guardar resultados intermedios 
		o las semillas aleatorias para facilitar la 
		reproducción de los datos.
	\item El experimentador debe ser imparcial y juzgar de 
		igual manera un algoritmo que otro, realizar 
		documentación ...
	\item Análisis estadístico: las afirmaciones y preguntas 
		que se realicen debe sostenerse estadísticamente 
		hablando. Es aconsejable un análisis visual para 
		exponer los datos obtenidos.
	\end{enumerate}
	
	

\subsection{Medidas del rendimiento}
	
	El rendimiento de un algoritmo sobre un problema puede 
depender del criterio que los investigadores deseen optimizar 
con lo que el concepto de la \textit{calidad} de un algoritmo 
depende según el problema al que se aplique. Por tanto, 
consideraremos el problema como un problema multiobjetivo. 
Una posible forma de abordarlo es condensar varias medidas en 
una única con la que posteriormente realizar las 
comparaciones necesarias a través de test estadísticos. Se 
presenta a continuación un resumen de las medidas comúnmente 
utilizadas en el campo del aprendizaje automático tanto para 
problemas de clasificación como de regresión, indicando 
ventajas e inconvenientes que nos ayuden a decidir aquella 
medida que mejor se ajuste a nuestro problema.
	
\subsubsection*{Medidas para problemas de clasificación}

	Las medidas utilizadas habitualmente para problemas de 
clasificación están basadas en la matriz de confusión 
(~\ref{tab:matrizconfusion}). Haremos un listado de aquellas 
más utilizadas sin entrar en las medidas diseñadas \textit{ad 
hoc} para problemas concretos. Suponemos que nos encontramos 
frente a un problema de clasificación binaria, aunque es 
posible extender la matriz de confusión y las medidas para 
problemas de clasificación con un mayor número de clases.
Notaremos por $x, \bar{x}$ la clase positiva y la negativa
respectivamente de nuestro problema, mientras que $y,\bar{y}$
se corresponden con las asignaciones a la clase positiva
y negativa respectivamente. Por tanto, notaremos por $TP$ a
los verdaderos positivos, por $TN$ a los verdaderos negativos,
$FP$ a los falsos positivos y $FN$ a los falsos negativos.

\begin{table}[H]
\centering
\begin{tabular}{cccc}
\cline{1-3}
\multicolumn{1}{|c|}{}           & \multicolumn{2}{c|}{Predicción del clasificador}      &                 \\ \cline{1-3}
\multicolumn{1}{|c|}{Clase real} & \multicolumn{1}{c|}{$y$}    & \multicolumn{1}{c|}{$\bar{y}$}    &                 \\ \cline{1-3}
\multicolumn{1}{|c|}{$x$}          & \multicolumn{1}{c|}{$TP$} & \multicolumn{1}{c|}{$FN$} & $N^+ = TP + FN$ \\ \cline{1-3}
\multicolumn{1}{|c|}{$\bar{x}$}          & \multicolumn{1}{c|}{$FP$} & \multicolumn{1}{c|}{$TN$} & $N^- = FP + TN$ \\ \cline{1-3}
                                & $\hat{N}^+ = TP + FP$     & $\hat{N}^+ = FN + TN$     &   $N = N^- + N^+$             
\end{tabular}
\caption{Matriz de confusión}
\label{tab:matrizconfusion}
\end{table} 

En función de esta matriz se definen las siguientes
medidas.
	
\begin{description}
	\item[Error de clasificación ($\varepsilon$),] 
		$\varepsilon = \frac{FP+FN}{N}$
	\item[Exactitud o \textit{Accuracy}(Acc),] 
		$Acc=\frac{TP+TN}{N}$
	\item[Exhaustividad o \textit{Recall}(r),] $r=\frac{TP}{N^+}$
	\item[Especificidad o \textit{Specificity}(sp),] 
		$sp=\frac{TN}{N^-}$
	\item[Precisión (pr),] 
		$pr=\frac{TP}{\hat{N}^+}$
	\item[Tasa falsos positivos(fpr)] $fpr=1-sp$
	\item[Tasa falsos negativos(fnr)] $fnr=1-r$
	\item[Valor negativo predicho(npv)] 
		$npv=\frac{TN}{\hat{N}^-}$
	\item[Estadístico kappa ($\kappa$)] El propósito de esta 
		medida es medir el grado de acuerdo entre dos 
		observadores, o, en nuestro contexto, compensar los 
		valores aleatorios además de minimizar el efecto de 
		un dispar número de ejemplos en cada clase. 
	\[ 
	\kappa = \frac{N(TP + TN) - N^+\hat{N}^+ - N^-\hat{N}^- }
				  {N^2 - N^+\hat{N}^+ - N^-\hat{N}^- }
	\]
\end{description}
	
	Las medidas de exactitud y el error de 
clasificación son comúnmente usadas debido a que representa 
un concepto claro y son fáciles de calcular.  Es necesario 
indicar que estas medidas son aconsejables cuando las dos 
clases se dan con una frecuencia similar y son igualmente 
relevantes para nuestro problema. Otras medidas como 
\textit{recall}, \textit{specificity} o \textit{precision} 
son más frecuentes en dominios como el de recuperación de 
información y aquellos donde o bien una clase es más 
relevante, o sólo existen unos pocos individuos de 
una clase.\\
	Para encontrar un equilibrio entre la detección de 
elementos de la muestra de ambas clases se han propuesto en 
la literatura (por ejemplo por Santafé, 
\cite{DBLP:journals/air/SantafeIL15}) algunas aproximaciones 
basadas en las anteriores medidas:
	
	\begin{description}
		\item[Compensación entre exhaustividad y 
especificidad.] El método más popular para encontrar 
este equilibrio es el análisis ROC, que consta de una 
representación de $r$ con respecto a $fpr$. Una curva ROC 
puede resumirse en el valor AUC (\textit{area under the ROC 
curve})
		\item[Compensación entre exhaustividad y 
precisión.] Depende del parámetro $\beta$ dado por el 
usuario :
		\[ F_\beta = \frac{(\beta^2+1) pr \cdot r}
						  {\beta^2 pr + r}	\]
	\end{description}

	Cuando en un problema de clasificación supervisada usamos 
un modelo probabilístico obtenemos la probabilidad de 
pertenencia a una de las clases, con lo que podemos realizar 
la asignación variando el umbral. Existen medidas que tratan 
de recoger el grado de proximidad entre esta asignación de 
probabilidad y la distribución real. La más conocida es la 
medida Brier ($BS$),
	\[ BS = \frac{1}{N} \sum\limits_{i=1}^N 
		\sum\limits_{j \in \{c^+,c^-\}}
			\left( P \left( C=j|\mathbf{x}^{(i)} \right) -
			\delta_{j,c^{(i)}} \right) ^2,\]
	donde $P\left(C=j|\mathbf{x}^{(i)}\right)$ es la 
probabilidad asignada por el clasificador a la instancia 
$i$-ésima para la clase $j$, $\delta$ es la delta de 
Kronecker y $c^{(i)}$ es la clase de la instancia 
$i$-ésima.\\
	Hasta ahora las medidas le han dado el mismo peso a los 
errores cometidos en ambos sentidos (asignar a la clase 
positiva una instancia negativa o viceversa), sin embargo se 
pueden incluir \textit{funciones de pérdida} ($\mathcal{L}$) 
según el coste relativo de cada tipo de fallo. Se 
puede considerar que en las medidas anteriores se ha usado 
la función de pérdida 
$\mathcal{L}(x,y) = 1 - \delta_{x,y} $, donde $x$ es la clase 
asignada e $y$ la clase real.
	
\subsubsection*{Medidas de interpretabilidad} 
	En algunos modelos y clasificadores se obtienen reglas 
para la clasificación de nuevos ejemplos. Por ello se valora 
positivamente que estas reglas sean entendibles para el 
usuario. 
	\begin{description}
	\item[Tamaño] Número de reglas que componen el modelo. 
		A menor tamaño, mayor interpretabilidad. 
		$\textit{Tamaño} = n_R$
	\item[Número de antecedentes (ANT)] Para $R_i$ una regla 
		de la forma $\textit{Condi-}$ ${ciones } \rightarrow 
		\textit{ Clase}$ donde las condiciones tienen la 
		forma $(Antecedente_1 \wedge \dots \wedge 
		Antecedente_k)$, definimos $Ant(R_i) = k$ y entonces 
		\[ANT = \frac{1}{n_R} 
				\sum\limits_{i=1}^{n_R} Ant(R_i).\]
	\end{description}


\subsubsection*{Medidas gráficas}

	Prati \textit{et al.} \cite{DBLP:journals/tkde/PratiBM11} afirman que en ocasiones es 
recomendable utilizar una representación gráfica del 
rendimiento ya que evitamos así pérdida de información al 
condensar toda la información en un escalar. En este artículo 
se presenta un conjunto de representaciones gráficas según el 
tipo de variable predicha: clases, ranking o probabilidad de 
pertenencia a una clase.
	
\paragraph{Predicción de clases}
	\begin{description}
		\item[Gráficas ROC] Como se ha mencionado 
		anteriormente, es una representación del la medida 
		\textit{tpr} frente a \textit{fpr}. Un clasificador 
		será mejor considerado cuanto más arriba y a la 
		izquierda se sitúe en el gráfico. 
		\item[Líneas de coste] Se trata de una modificación 
		de las gráficas ROC para incluir el coste asociado a 
		cada tipo de fallo. Se define el coste esperado para 
		un clasificador como 
		\[ EC = \sum\limits_{X \in \{x, \bar{x}\}} 
		 		\sum\limits_{X \in \{x, \bar{x}\}} 
		 			P(X,Y) \mathcal{L}(X,Y),
		 \]
con $P(X,Y)$ la probabilidad de clasificar como $Y$ un
elemento de la clase $X$ y $\mathcal{L}(X,Y)$ la
función de coste asociado a considerar $Y$ un elemento
de la clase $X$.
Entonces, se definen $NormEC = \frac{EC}{maxEC}$, $PC = 
\frac{p(x) \mathcal{L}(\bar{x},y)}{maxEC}$, donde $maxEC 
= p(x)\mathcal{L}(\bar{x},y)+p(\bar{x})\mathcal{L}(x,
\bar{y})$.  La gráfica representa la $PC$ frente a 
$NormEC$. Ahora cada clasificador, representado por un 
punto en la gráfica ROC, es una línea. El clasificador 
representado por el punto $(0,0)$ en la gráfica ROC (asigna a 
todos los puntos la clase negativa) ahora se representa 
mediante la recta $NormEC=PC$ y el punto $(1,1)$ mediante la 
recta $NormEC= 1-PC$. Dependiendo del coste relativo de la 
clase positiva ($PC$), será el mejor clasificador aquel con 
menor $NormEC$.
	\end{description}

\paragraph{Predicción de \textit{ranking}} 	
	\begin{description}
		\item[Curvas ROC] Se traslada el problema a uno de 
		clasificación mediante la asignación a la clase 
		positiva del \textit{top} $n\%$. Se varía el 
		porcentaje del $0\%$ al $100\%$ para obtener la curva 
		ROC.
		\item[Curvas de coste] Se trata de la transformación 
		análoga a las curvas ROC para las líneas de coste.
		\item[Curvas \textit{precision-recall}] Usadas 
		habitualmente en el ámbito de la recuperación de 
		información. Representan $precision$ frente a 
		$recall$.
		\item[Gráfica \textit{lift}] Similar a la gráfica 
		ROC, representa $tpr$ frente a la probabilidad de 
		asignar a un individuo la clase positiva. 
		\item[Gráfica ROI] Similar a las gráficas 
		\textit{lift}, pero representando en el eje $Y$ el 
		beneficio esperado 
		\[	TEP = N \sum\limits_{X \in \{x,\bar{x}\}}
					\sum\limits_{Y \in \{y,\bar{y}\}}
						p(X,Y) \mathcal{L}(X,Y) p(X) \]
	\end{description}

\paragraph{Predicción de probabilidad de pertenencia a una clase} 	
	\begin{description}
		\item[\textit{Reliability diagram}] Contiene dos
		gráficas. Una es $p(x|j)$ según $j$, que mide el 
		grado en que concuerdan las predicciones con la
		frecuencia de los casos positivos. La segunda gráfica 
		se corresponde con $p(y_j)$ según $j$, identificada 
		con la confianza en la predicción.
		\item[Diagrama de atributos] Para evaluar las mejoras 
		relativas de un sistema de predicción con respecto a 
		uno de referencia se define la $BSS$ (\textit{Brier 
		skill score})
			\[BSS = 1 - \frac{BS}{BS_{ref}} \]
		El diagrama de atributos es un refinamiento del
		\textit{reliability diagram}, al que se le añaden 
		tres líneas: 
			\begin{description}
			\item[Línea de no resolución:] línea horizontal 
			donde $p(x|j)=p(x)$.
			\item[Línea de máxima resolución:] línea vertical 
			que intersecta a la diagonal en el mismo punto 
			que la línea de no resolución.
			\item[Clasificador de referencia:] Bisectriz 
			entre la diagonal principal y la línea de no 
			resolución. Representa los puntos cuya precisión 
			(\textit{accuracy}) es menor que en el modelo de 
			referencia.
			\end{description}
		\item[\textit{Discrimination diagram}] Gráfica de 
		$P(Y|X)$ según $y_j$. Contiene las curvas de 
		$p(y_j|x)$ y de $p(y_j|\bar{x})$. En el mejor de los 
		casos, estas dos curvas no se solaparían y podríamos 
		saber la clase según $y_j$.
	\end{description}
	
	
\subsection*{Medidas para problemas de regresión}

	Las medidas para un problema de regresión están basadas 
en la distancia entre el valor real de la instancia y el 
valor predicho. Consideraremos que calculamos el error de la 
predicción del algoritmo $f$ aplicado a los datos $x_1, 
\dots, x_N$ con valor $y_1, \dots, y_N$ respectivamente.
	
\begin{description}
	\item[\textit{Mean squared error} (MSE)]: Error 
	cuadrático medio
		\[ MSE(f) = \frac{1}{N} \sum\limits_{i=1}^N
								(y_i - f(x_i))^2	\]
	\item[\textit{Root mean squared error} (RMSE)]: Raíz del 
	error cuadrático medio
		\[ RMSE(f) = \sqrt{\frac{1}{N} 
		\sum\limits_{i=1}^N (y_i - f(x_i))^2}	\]
		
	\item[\textit{Mean absolute error} (MAD) ]: Error 
	absoluto medio
		\[ MAD(f) = \frac{1}{N} \sum\limits_{i=1}^N
								|y_i - f(x_i)|	\]
\end{description}

	Las medidas $MSE$ y $RMSE$ son más utilizadas debido a la 
penalización de los errores mayores. 

\subsection{Validación cruzada y remuestreo}
\label{sssec:CV}
	
	Para la repetición de los experimentos, necesitamos un 
número de conjuntos de entrenamiento y validación del 
conjunto de datos disponible. Si el conjunto es 
suficientemente grande, podemos dividir aleatoriamente el 
conjunto en $K$ partes, y de cada parte seleccionar la mitad 
para entrenamiento y la otra mitad para test. $K$ suele ser 
10 ó 30. Sin embargo, los conjuntos de datos no suelen ser 
tan grandes. Por tanto, lo que habitualmente se realiza es 
utilizar varias veces los mismos datos pero de formas 
distintas, lo que se conoce como \textbf{validación cruzada} 
(\textit{cross-validation}, CV). Otro concepto a tener en 
cuenta es el de \textbf{estratificación}. Consiste en 
mantener la proporción de las diferentes clases en cada una 
de las particiones realizadas en la base de datos. 
	
\paragraph{Validación cruzada con $K$-fold} Se divide el 
conjunto de datos $\mathcal{X}$ en $K$ partes aleatoriamente, 
$\mathcal{X} = \mathcal{X}_1 \cup \dots \cup \mathcal{X}_K$. 
Para formar cada par de datos de entrenamiento y validación, 
$(\mathcal{T}, \mathcal{V})$, se mantiene una de las $K$ 
partes y se unen las demás partes como conjunto de 
entrenamiento
\begin{align*}
	\mathcal{V}_i &= \mathcal{X}_i \\
	\mathcal{T}_i &= \underset{j=1,\dots,K; j\neq i}
							\bigcup \mathcal{X}_j
\end{align*} 
	El problema de este método es que el conjunto de 
validación es pequeño (lo que lleva a una mayor variabilidad 
en el error). Además, los conjuntos de entrenamiento se 
solapan considerablemente. $K$ es habitualmente 10 ó 30. 
Conforme $K$ crece, se incrementa la robustez al afectar 
menos cada dato concreto. El caso $K = n-1$, se conoce como 
\textit{leave-one-out}. 

\paragraph{$5 \times 2$ CV} Otra propuesta consiste en 
dividir $\mathcal{X}$ en dos partes, utilizar una de estas 
partes como datos de entrenamiento y la otra como test e 
invertir los roles. Para obtener el segundo \textit{fold}, se 
realiza otra partición aleatoria de $\mathcal{X}$ y se repite 
el proceso. Lo habitual es que se realice cinco veces, 
obteniendo así diez parejas $(\mathcal{T}, \mathcal{V})$. 
Aunque podría realizarse más veces, al realizarse más veces 
los conjuntos comparten muchas instancias y por lo tanto los 
errores son muy dependientes entre sí, con lo que no se 
aporta nueva información. Si se realiza menos de cinco veces, 
se obtienen pocas muestras y es más complejo comprobar las 
hipótesis.

\paragraph{\textit{Bootstrapping}} Para generar varias 
muestras a partir de una única se seleccionan instancias de 
la original con reemplazamiento. Puede solaparse más que con 
CV, con lo que las estimaciones resultan más dependientes. 
Por ello, se recomienda especialmente para conjuntos de datos 
muy pequeños en los que se necesiten mayores muestras. 

	Al realizar \textit{bootstrap}, se seleccionan $N$ 
instancias de un conjunto de tamaño $N$ con reemplazamiento. 
El conjunto original se usa como conjunto de validación. La 
probabilidad de no seleccionar una instancia es $1-1/N$, con 
lo que la probabilidad de no seleccionarla tras $N$ 
elecciones es $\left(1- \frac{1}{N}\right)^N \approx e^{-1} 
\approx 0.368$. Esto significa que el conjunto de 
entrenamiento contiene aproximadamente un $63.2\%$ de los 
datos. Para obtener una mejor estimación del error, se 
propone repetir el proceso y observar el comportamiento 
medio.
	
\subsection{Test de hipótesis}

	Una vez se han extraído los datos, pretendemos obtener
las conclusiones de manera estadística. Para ello, la 
forma habitual de proceder es mediante los test de hipótesis.
En la realización de los test de hipótesis, el 
procedimiento es el siguiente: se define un estadístico que 
sigue una distribución conocida en el caso de que se cumpla 
la hipótesis realizada. Entonces si el estadístico calculado 
de la muestra tiene poca probabilidad de obtenerse de la 
distribución, se rechaza la hipótesis; en caso contrario, no. 
\\
	Surgen entonces las cuestiones: ¿Se puede justificar los 
resultados estadísticamente o se obtienen por azar? ¿Son los 
conjuntos de datos representativos para el problema? Estas 
preguntas no se pueden responder de forma exhaustiva, es la 
labor de los test estadísticos reunir los datos disponibles 
para justificar la consistencia de las conclusiones. \\
	A la hora de realizar los test estadísticos podemos 
realizar las siguientes comparaciones:
	\begin{enumerate}
	\item Comparación de dos algoritmos en un dominio: $t$-
		test (paramétrico), test de McNemar (no paramétrico).
	\item Comparación de múltiples algoritmos en un dominio: 
		\textit{Signed-Rank test} de Wilcoxon y test de signo 
		(no paramétricos).
	\item Comparación de múltiples algoritmos en varios 
		dominios: ANOVA (paramétrico), test de Friedman o 
		basados en permutaciones (no paramétricos). 
	\end{enumerate}

	Nótese que el test de \textit{Signed-Rank} de Wilcoxon 
para muestras emparejadas puede aplicarse tanto para la 
comparación de dos como entre varios algoritmos en un 
dominio. Los test para varios algoritmos en varias bases de 
datos necesitan un test posterior en caso de rechazar la 
hipótesis de que todos los algoritmos se comporten de igual 
manera. Ejemplos de estos test son el de Tukey, de 
Bonferroni-Dunn o de Nemenyi. 


%
%\documentclass[11pt,leqno]{book}
%\usepackage[spanish,activeacute]{babel}
%\usepackage[utf8]{inputenc}
%\usepackage{enumerate}
%
%\begin{document}


\chapter{Desarrollo matemático}

\section{Conceptos básicos de inferencia}

\section{Test paramétricos}

\section{Test no paramétricos}

	\subsection{Test basados en permutaciones}

\section{Test bayesianos}	
%\end{document}
%
\chapter{Desarrollo informático}

	En
%
%\documentclass[11pt,leqno]{book}
%\usepackage[spanish,activeacute]{babel}
%\usepackage[utf8]{inputenc}
%\usepackage{enumerate}
%
%\begin{document}


\chapter{Conclusiones}

\section*{Críticas y sugerencias}

\section*{Conclusiones y trabajos futuros}
%\end{document}
%
%%\nocite{*}
\begin{thebibliography}{99}

\bibitem{JAPSHA11}
  N. Japkowicz, M. Shah,
  \emph{Evaluating Learning Algorithms - A Classification Perspective},
  Cambridge University Press,
  2011.

\bibitem{SANTAFE15}
  G. Santafé, I. Inza, J. A. Lozano,
  \emph{Dealing with the evaluation of supervised classification algorithms},
  Springer Science - AIRE,
  2015.
  
\bibitem{PRATI11}
  R. C. Prati, G. E.A.P.A. Batista, M. C. Monard,
  \emph{A Survey on Graphical Methods for Classification Predictive Performance Evaluation},
  IEEE TRANSACTIONS ON KNOWLEDGE AND DATA ENGINEERING, VOL. 23, NO. 11,
  2011

\bibitem{DICCHA11}
  J. Dickinson Gibbons, S. Chakraborti,
  \emph{Nonparametric Statistical Inference},
  Fifth edition,
  Chapman and Hall/CRC,
  2011
	
\bibitem{SHESKIN11}
  D. J. Sheskin,
  \emph{Handbook of Parametric and Nonparametric Statistical Procedures},
  Fourth edition,
  Chapman and Hall/CRC,
  2007
  
\bibitem{PESSAL10}
	F. Pesarin, L. Salmaso,
	\emph{Permutation Test for Complex Data}
	WILEY, 
	2010
  
\bibitem{KANUNGO00}
	T. Kanungo, R. M. Haralick, H. S. Baird, W. Stuezle, D. Madigan,
  \emph{A Statistical, Nonparametric Methodology for Document Degradation Model Validation},
  IEEE Transactions on Pattern Analysis and Machine Intelligence,
  VOL. 22, NO. 11,
  2000
  
\bibitem{CHENGCHEN03}
	Xiuzhen Cheng, Dechang Chen
  \emph{Statistical Comparisons of Multiple Classifiers},
  2003
  
\bibitem{DEMSAR06}
	J. Dem\v{s}ar
	\emph{Statistical Comparisons of Classifiers over Multiple Data Sets},
	Journal of Machine Learning Research 7,
	2006
	
	
\bibitem{GARCIAHERRERA08}
	S. García, F. Herrera,
	\emph{An Extension on ``Statistical Comparisons of Classifiers over Multiple Data Sets'' for all Pairwise Comparisons},
	Journal of Machine Learning Research 9,
	2008
	
\bibitem{GARCIA09}
	S. García, D. Molina, M. Lozano, F. Herrera
	\emph{A study on the use of non-parametric test for analyzing the evolutionary algorithms' behaviour: a case study on the CEC'2005 Special Session on Real Parameter Optimization},
	Journal of Heuristics,
	2009

\bibitem{GARCIA08}
	S. García, A. Fernández, J. Luengo, F. Herrera
	\emph{A study of statistical techniques and performance measures for genetics-based machine learning: accuracy and interpretability},
	Springer-Verlag,
	2008
	
\bibitem{OTERO13}
	J. Otero, L. Sánchez, I. Couso, A. Palacios.,
	\emph{Bootstrap analysis of multiple repetitions of experiments using an interval-valued multiple comparison procedure}
	Elsevier, Journal of Compuyer and System Sciences,
	NO. 80,
	2013

\bibitem{ALPAYDIN10}
	E. Alpaydin,
	\emph{Introduction to Machine Learning},
	2a. Edición, 
	The MIT Press,
	2010

\bibitem{HOLLANDER14}
	M. Hollander, D.A. Wolfe, E. Chicken,
	\emph{Nonparametric Statistical Methods},
	3a. Edición,
	Wiley,
	2014
	
\end{thebibliography}
%
%\appendix
%\input{apendices/manual_usuario/manual_usuario}
%%\input{apendices/paper/paper}
%\input{glosario/entradas_glosario}
% \addcontentsline{toc}{chapter}{Glosario}
% \printglossary
\chapter*{}
\thispagestyle{empty}

\end{document}
