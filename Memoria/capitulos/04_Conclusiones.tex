%\documentclass[11pt,leqno]{book}
%\usepackage[spanish,activeacute]{babel}
%\usepackage[utf8]{inputenc}
%\usepackage{enumerate}
%
%\begin{document}


\chapter*{Conclusiones y trabajos futuros}\addcontentsline{toc}{chapter}{Conclusiones}


	Las conclusiones obtenidas con la realización de este trabajo son principalmente que la estadística es una herramienta fundamental para la comparación entre el rendimiento de algoritmos en el ámbito del aprendizaje automático y que la comparación de estos algoritmos debe 
realizarse de manera rigurosa para poder obtener resultados confiables. Se han observado cómo existen diversos métodos para la realización de los test estadísticos, los cuales 
deben usarse bajo unas ciertas condiciones que, en caso 
de no cumplirse, conllevan una pérdida de potencia en la comparación, como es el caso de los test paramétricos. Por otra parte, frente a los test de hipótesis nula, muchas veces malinterpretados, surgen los test bayesianos como solución a algunos de estos problemas.\\
Otra conclusión a destacar es la importancia del manejo de
los conjuntos de datos, la separación en conjuntos de entrenamiento y test y la validación cruzada.\\
Como trabajo futuro, queda pendiente la integración completa del paquete de Java y la profundización en nuevos test para la comparación de algoritmos. 


%\end{document}