 
 	La principal ventaja de los test no paramétricos es que las hipótesis son más generales con lo que se pueden aplicar en un mayor número de problemas. Una de las condiciones habituales es la continuidad, aunque hay otras condiciones más estrictas como que la población sea simétrica para según qué test. Esto repercute en que se le pueden aplicar funciones a las muestras obtenidas para la realización de los test, a diferencia de en los test paramétricos dado que las muestras deben generalmente provenir de una población de forma conocida.\\
 	
 	Hay test de hipótesis que no están relacionados con valores de parámetros (a diferencia de los paramétricos). Son más simples de aplicar, las matemáticas, menos sofisticadas y basadas en la combinatoria, están relacionadas con las propiedades usadas en el proceso inductivo. Los libros de recetas no son necesarios, pues con la mera definición del test queda suficientemente claro cómo aplicar el test no paramétrico. Además, las distribuciones asintóticas son distribuciones conocidas como la normal o la chi cuadrado. Al relajar las condiciones sobre los datos de entrada, es menos sensible al \textit{dirty-data}, datos con errores usados en el entrenamiento del clasificador. Esto implica una mayor robustez en los test no paramétricos.\\
 	
 	Correctamente aplicados, los test paramétricos, al disponer de mayor información tienen una mayor potencia, sin embargo, cuando se disponen de menos datos, y por tanto es más difícil que se den las condiciones de los test paramétricos, la potencia es similar.