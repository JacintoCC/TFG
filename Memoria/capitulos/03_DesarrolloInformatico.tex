\part{Desarrollo informático}
\label{part:informatica}

\chapter{Revisión del Estado del Arte en la Aplicación de test en la evaluación de algoritmos de aprendizaje automático}

\section{Test paramétricos}


\subsection{5x2 CV $t$-test}
	
	En la introducción (\ref{sec:CV}), se describe el 
proceso de validación cruzada, especialmente útil cuando se
disponen de pocos datos y queremos mantener un equilibrio
entre el tamaño del conjunto de entrenamiento y el de test.
Para esta situación, se propone la siguiente variación 
del $t$-test. Se considera $d_i^{(j)}$, para $j=1,2$, $i=
1,\dots,5$, la diferencia entre el rendimiento de los dos
clasificadores para la repetición $i$ y el \textit{fold} $j$.
Se considera la media para cada repetición $\bar{d}_i = 
(d_i^{(1)} + d_i^{(2)})/2$ y la varianza estimada
$s_i^2 = (d_i^{(1)}-\bar{d_i})^2 + (d_i^{(2)}-\bar{d_i})^2$. 
Bajo la hipótesis nula de que los dos clasificadores tienen 
el mismo rendimiento, $d_i^{(j)}$ puede considerarse 
aproximadamente normal . Si suponemos $d_i^{(1)}$ y 
$d_i^{(2)}$ independientes y normales (lo cual no es
cierto porque los conjuntos de test y de entrenamiento
no se han obtenido de manera independiente), entonces
$s_i^2/\sigma^2$ sigue una distribución $\chi^2$ con un
grado de libertad. Si suponemos las $s_i^2$ independientes,
su suma $M = \frac{\sum\limits_{i=1}^5 s_i^2}{\sigma^2}
\sim \chi^2_5$ y 
\begin{equation}
	\label{eq:5x2CVt}
	t = \frac{d_1^{(1)}/\sigma}{\sqrt{M/5}} = 
		\frac{d_1^{(1)}}
			{\sqrt{\sum\limits_{i=1}^5 s_i^2/5}}
		\sim t_5.
\end{equation}

\paragraph{Modificación} Nótese que en la 
ecuación~\ref{eq:5x2CVt} el valor $d_1^{(1)}$ podría ser
ocupado por cualquier estadístico $d_i^{(j)}$. Alpaydin
propuso combinar estos valores de la forma:
	\[
		N = \frac{\sum\limits_{i=1}^5
				\sum\limits_{j=1}^2
					\left( d_i^{(j)} \right)^2}
				{\sigma^2}
			\sim \chi_{10}^2.
	\]
	Sustituyendo esto en \ref{eq:5x2CVt}, tenemos
un estadístico que es el cociente de dos v.a. que siguen 
una $\chi^2$, con lo que definimos la v.a. $f$, que seguirá
una distribución $F$ de Snedecor con 10 y 5 grados de
libertad:
	\[ 
		f = \frac{N/10}{M/5} \sim F_{10,5}.
	\]
	
	
\section{Test no paramétricos}
 
\paragraph{Validación de un modelo de degradación de 
documentos} \cite{DBLP:journals/pami/KanungoHBSM00} En este artículo se presenta un 
modelo de la degradación en documentos producida por la 
digitalización de documentos impresos o la realización de 
fotocopias. El problema estadístico que se define es el 
siguiente. Dadas las muestras $x_1, \dots, x_N$ e $y_1, 
\dots, y_M$ de caracteres degradados de un mismo carácter de 
un documento y generados de manera artificial, 
respectivamente, realizar el test sobre la hipótesis nula de 
que ambos  provienen de la misma población para un valor de 
significación $\alpha$. Se aplica el algoritmo 
(\ref{alg:CMC-digitalizacion}) para realizar el test basado 
en permutaciones calculando el $p$-valor utilizando el 
algoritmo~\ref{alg:CMC-pvalue}.

\begin{algorithm}
	\caption{Test basado en permutaciones para validación en 
			modelo de degradación}
	\label{alg:CMC-digitalizacion}
	\begin{algorithmic}[1]
	\REQUIRE
		\begin{enumerate}[a]
		\item Datos reales $X = \{ x_1, \dots, x_N \}$
		\item Datos generados $Y = \{ y_1, \dots, y_M \}$
		\item Función distancia entre conjuntos $\rho(X,Y)$
		\item Función distancia entre caracteres $\delta(x,y)$
		\item Tamaño máximo del test $\alpha$
		\end{enumerate}
		\STATE Calcular $d_0 = \rho(X,Y)$
		\STATE Crear una muestra 
			$Z= \{x_1, \dots, x_N, y_1, \dots, y_M \}$.
		\FOR { $i \in \{1, \dots, K\}$ }
			\STATE Realizar una permutación de $Z$.
			\STATE Particionar $Z$ en $X'$ e $Y'$ tales que 
				$X' = \{ z_1, \dots, z_N \}$, $Y' = 
				\{ z_{N+1}, \dots, z_{N+M} \} $.
			\STATE Calcular $d_i = \rho(X,Y)$
		\ENDFOR
		\STATE Calcular la distribución empírica de las $d_i$
		\STATE Calcular el $p$-valor: 
			$\alpha_0 = P(d \geq d_0)$
		\STATE Rechazar la hipótesis nula si 
			$\alpha_0 \leq \alpha$
	\end{algorithmic}
\end{algorithm}

	
\paragraph{Comparación estadística de varios clasificadores} 
\cite{DBLP:conf/mlmta/ChenC03} En este artículo se describe un método 
para realizar la comparación estadística de varios 
clasificadores sobre una misma base de datos a través del 
test de Cochran, una generalización del test de McNemar. La 
aplicación de un test no paramétrico en lugar de uno 
paramétrico como ANOVA se justifica en el artículo debido a 
que no se puede asegurar la independencia entre las 
instancias clasificadas por un mismo clasificador, con lo que 
queda en entredicho el desempeño del test ANOVA, por ejemplo. 
Si la hipótesis nula, $H_0: \theta_1 = \dots = \theta_n$ 
(todos los algoritmos tienen un rendimiento equivalente), es 
cierta, no es necesario ningún análisis posterior. Si es 
rechazada, se pretenden realizar múltiples comparaciones para 
encontrar los algoritmos con mejor rendimiento. Para ello se 
describen los métodos de Bonferroni y Scheffé. El primero 
parte de una familia de $g$ contrastes, mientras que el 
segundo considera la familia $L$ de todos los contrastes 
posibles. Por ello, para el primero se deben prefijar las 
comparaciones a realizar, mientras que para el segundo esto 
se puede realizar \textit{a posteriori} según los datos 
obtenidos. Para escoger entre un método u otro, se recomienda 
escoger el primero cuando $Z_{1-\alpha/2g}$ (percentil $(1-
\alpha/2g)100$ de una distribución normal estándar) es menor 
que $\sqrt{\chi^2_{m-1,1-\alpha}}$ (donde $m$ es el número de 
algoritmos en la comparación) debido a que en este caso el 
método de Bonferroni tiene una mayor potencia. Escogeremos el 
método de Scheffé en caso contrario.

\paragraph{Comparación estadística de clasificadores en 
varios conjuntos de datos} \cite{DBLP:journals/jmlr/Demsar06} De la comparación 
utilizando un único conjunto de datos se pasa en este 
artículo a presentar un método para comparar el rendimento 
sobre varios conjuntos de datos utilizando test no 
paramétricos.\\
	Se incluyen en una primera sección las opciones para 
realizar las comparaciones entre dos clasificadores para 
varias conjuntos de datos, evaluando sus ventajas e 
inconvenientes:
	\begin{enumerate}
	\item Media entre los conjuntos de datos: Arroja poca 
		información poco valiosa al combinar puntuaciones 
		obtenidas en distintos dominios.
	\item $t$-test emparejados: El estadístico de este test 
		paramétrico es similar al usado al realizar la media 
		entre los conjuntos de datos. Además, se necesita que 
		la diferencia siga una distribución normal, además de 
		que los datos disponibles suelen ser reducidos. 
	\item Test de signo: Como se ha dicho en el desarrollo de 
		los test, no tiene en cuenta la magnitud de la 
		diferencia.
	\item Test de rangos con signo de Wilcoxon: Más seguro 
		que el $t$-test al no suponer la distribución normal. 
		Si no se dan las circunstancias adecuadas, este test 
		puede tener mayor potencia que el $t$-test.
	\end{enumerate}
	
	En cuanto a la comparación entre varios clasificadores, 
se incluye una breve descripción del método ANOVA y una 
descripción más extensa del test de Friedman. Se incluye el 
estadístico $F_F = \frac{(n-1) \chi^2_F}{n(k-1) - \chi^2_F}$ 
que sigue la distribución $F$ con $k-1$ y $(k-1)(n-1)$ grados 
de libertad. Aunque el test ANOVA tiene mayor potencia cuando 
las condiciones se cumplen, se incluye un experimento que 
muestra que apenas hay diferencias si éstas no se dan.\\

	Una vez se ha rechazado la hipótesis nula se describen 
los test \textit{post-hoc} de Nemenyi, Bonferroni-Dunn, Holm, 
Hochberg y Hommel.
	
	
\paragraph{Extensión sobre los procedimientos 
\textit{post-hoc}} \cite{garcia2008extension} Este artículo 
extiende al anterior prestando una mayor atención a la 
descripción de los procedimientos para la comparación entre 
algoritmos una vez rechazada la hipótesis nula. Se introducen
los procedimientos de Nemenyi y Holm. Debido a que las 
posibles hipótesis sobre la comparación  por parejas están 
relacionadas y no se pueden dar todas las posibles 
combinaciones, se presentan los procedimientos de 
Schaffer estático y dinámico. Bergmann y Hommel propusieron 
un procedimiento basado en la idea de encontrar todas las 
hipótesis que no pudieran ser rechazadas. En un estudio 
comparativo sobre el rendimiento de cinco clasificadores 
sobre treinta conjuntos de datos se observa como el 
procedimiento de Bergmann y Hommel tiene una mayor 
potencia.\\

	Se aborda también el problema de calcular $p$-valores 
ajustados (APV) que tengan en cuenta los demás test 
realizados para poder utilizar directamente este valor 
ajustado. Se indican el cálculo para cada procedimiento 
\textit{post-hoc}. Tras realizar un nuevo estudio 
introduciendo el uso de APV, se descarta el test de Nemenyi 
debido a que es demasiado conservador. Se recomienda el 
procedimiento estático de Schaffer frente al del Holm al 
tener mayor potencia al usar las relaciones entre las 
hipótesis y una dificultad no muy superior. Debido a su coste 
computacional y su complejidad, se recomienda el método de 
Bergmann-Hommel sólo para cuando las diferencias entre 
clasificadores sean muy pequeñas y otros métodos no detecten 
diferencias significantivas.

	\paragraph{Estudio del uso de test no paramétricos} 
	\cite{DBLP:journals/heuristics/GarciaMLH09} En este artículo se realiza un estudio
sobre el uso de test no paramétricos para comparar el 
rendimiento de algoritmos evolutivos en la Sesión Especial de 
Optimización de parámetro real en el Congreso de Computación 
Evolutiva (CEC 2005). Los datos disponibles son los errores 
cometidos por los algoritmos a la hora de minimizar 25 
funciones de 10, 30 y 50 variables reales. En primer lugar se 
estudian las condiciones para realizar un test paramétrico 
($t$-test) llevando a cabo test de normalidad (test de 
Kolmogoro-Smirnov, test de Shapiro-Wilk y el de D'Agostino-
Pearson) y el test de Levene de homocedasticidad. Se 
pretende realizar en primer lugar la comparación entre dos de 
los algoritmos para comprobar si hay diferencias entre ellos. 
Sin embargo los test de normalidad rechazan para los 
resultados en casi todas las funciones a minimizar la 
hipótesis de normalidad y el test de Levene rechaza la 
hipótesis de la equivalencia de las varianzas, con lo que no 
se cumplen las condiciones para realizar el $t$-test. Si se 
realizan el $t$-test y el test no paramétrico de Wilcoxon se 
observa que los resultados son muy similares.\\
	A la hora de afrontar la comparación en múltiples 
problemas en lugar de problema por problema en este artículo 
se toma la media de las distintas ejecuciones. Al realizar 
los test de normalidad sobre las dos muestras de 25 elementos 
para cada algoritmo se descarta la normalidad. El $t$-test y 
el test de Wilcoxon no rechazan la hipótesis nula de que 
ambos algoritmos sean igual de buenos, sin embargo el $p$-
valor del test de Wilcoxon es bastante menor, lo que nos 
lleva a pensar que, debido a la ausencia de normalidad y de 
pocos datos y difíciles de incrementar (pues en este caso 
deberíamos añadir nuevas funciones y sus 
correspondientes evaluaciones para cada algoritmo), en estos 
casos es mejor utilizar test no paramétricos.\\
	Para realizar el análisis sobre todos los algoritmos se 
separan las funciones fáciles, aquellas en las que algún 
algoritmo alcanza el óptimo frente a las que ningún algoritmo 
llegó al óptimo. El test no paramétrico de Friedman y el de 
Iman-Davenport arrojan valores que nos llevan a rechazar la 
hipótesis nula tanto usando el grupo de funciones difíciles 
como usando el conjunto total de funciones.\\
	El estudio posterior se centra en comparar los algoritmos 
con aquel que consiguió un menor error medio, el 
\textit{G-CMA-ES}. Para ello se utilizan el test de 
Bonferroni-Dunn, el de Holm y el de Hochberg para los dos 
grupos de funciones. El test de Hochberg llega a encontrar 
diferencias entre el mejor algoritmo y todos los demás para 
$\alpha=0.1$. 
	
	\paragraph{Estudio estadístico sobre algoritmos 
genéticos: precisión e interpretabilidad} \cite{Garcia2008} En 
este artículo se realiza la comparación sobre algoritmos 
evolutivos utilizados en clasificación, comparando tanto la 
tasa de clasificación como el valor kappa de Cohen y la 
interpretabilidad de estos resultados. Al igual que en el 
artículo anterior, se realizan test de normalidad y 
homocedasticidad para comprobar si se dan las condiciones 
para aplicar un test paramétrico. Si las comparaciones por 
parejas de problemas se repiten crece el error asociado FWER 
(\textit{family-wise error rate}), esto es la probabilidad de 
cometer al menos un error en la familia de hipótesis. Para realizar 
comparaciones entre varios algoritmos y fijar el FWER de 
antemano se utilizan el test de Friedman y el de Iman-
Davenport y los test de Bonferroni-Dunn y el de Holm para 
buscar diferencias entre los algoritmos una vez descartada la 
hipótesis de que todos tienen la misma media. Se 
calculan también los $p$-valores ajustados.\\
	Para realizar el análisis de la interpretabilidad de los 
resultados se considera $\textit{Complejidad} = 
\textit{Tamaño} \cdot ANT$. Se realiza para este estudio 
únicamente un procedimiento para múltiples comparaciones, 
obteniendo diferencias significativas con el test de 
Bonferroni-Dunn y nivel de significación $\alpha=0.1$.
	
	\paragraph{Test avanzados para la comparación y análisis 
experimental de la potencia} \cite{DBLP:journals/isci/GarciaFLH10} En este
artículo se introducen con respecto a otros artículos 
previos los test de múltiples signos y la estimación del 
contraste basado en medianas como test básicos para realizar
entre múltiples clasificadores y conjuntos de datos como 
test básicos, y los más avanzados del test de Friedman
de rangos alineados (\textit{Friedman Aligned Ranks}, 
Friedman-AR) y el test de Quade. Además, para realizar las 
comparaciones  posteriores  una vez descartada la hipótesis 
de un igual  rendimiento entre todos los algoritmos, se 
incluyen a los  procedimientos \textit{post-hoc} los de 
Holland y Finner  dentro de  los métodos descendentes, el de 
Rom como método  ascendente y el procedimiento de Li, para 
rechazar hipótesis en dos pasos.\\
	Se realiza un estudio sobre la potencia de los test
de Friedman, de Friedman-AR y de Quade, utilizando 8 y 4
algoritmos sobre 10 y 6 conjuntos de datos respectivamente. 
Se observa que el test de Friedman-AR es más conservador
cuando el número de algoritmos es elevado. En este estudio
se observa que el test de Quade se comporta mejor, sin
embargo depende de los conjuntos de datos utilizados. Para
el estudio sobre los test \textit{post-hoc}, se utilizan
8 algoritmos sobre 10 conjuntos de datos seleccionados
de forma pseudo-aleatoria (regulando mediante un parámetro la 
probabilidad de seleccionar una base de datos según la 
diferencia de los algoritmos a comparar en ella). Los 
resultados obtenidos hacen desaconsejar el test de 
Bonferroni-Dunn, en un siguiente grupo estarían Holm, 
Hochberg, Hommel, Holland y Rom con una potencia similar
y los que mejores resultados obtienen son el test de Finner
y el test de Li, el cual ofrece mejores resultados 
cuando las diferencias entre el algoritmo de control y los 
demás sean grandes.

	
	\paragraph{Análisis \textit{bootstrap} de múltiples 
repeticiones de experimentos} \cite{Otero201488} En este artículo 
se presenta la realización de test basados en permutaciones 
para la comparación del rendimiento de algoritmos de 
clasificación, realizando un estudio para el que el test 
presentado tiene mayor potencia que los test de ANOVA o 
Friedman. El contexto para el que se presenta este método es 
el de la comparación entre algoritmos estocásticos (de ahí 
que se realice la repetición de la ejecución de los 
algoritmos con diferentes semillas) utilizando validación 
cruzada. Se propone usar intervalos para recoger parte de la 
variabilidad de los datos atribuible a la repetición del 
experimento. Se plantea el problema del uso de la media entre 
los distintos experimentos para evaluar un algoritmo debido a 
que los valores extremos tienen una influencia excesiva y se 
aboga por otros estadísticos como la mediana, el rango 
intercuartílico o la media restringida a los valores 
centrales.\\
	El primer test es un test basado en permutaciones. Parte 
de la muestra compuesta por $e_{adfr}$, que es el error del 
algoritmo $a$-ésimo, el conjunto de datos $d$, para el 
\textit{fold} $f$ y la repetición $r$. Se utiliza en este 
test como estadístico para evaluar cada algoritmo la media de 
las ejecuciones en todos los \textit{folds}, repeticiones y 
conjuntos de datos como estimador, utilizando el algoritmo ~
\ref{alg:CMC-pvalue}. Se rechazará la hipótesis nula (la 
equivalencia de las medias) si para algún $a$ la media 
inicial $\hat{e}_a$ se encuentra en las colas de la 
distribución formada por los valores de las permutaciones $
\hat{e}_a^*$ consistente en la cdf muestral:
	\[ 
		\hat{F}_{a \cdot \cdot \cdot}(x) =
			\frac{1}{n_d n_r n_f}
			\sum\limits_{i=1}^{n_d}
				\sum\limits_{j=1}^{n_f}
					\sum\limits_{k=1}^{n_r}
						\mathbb{I}[ e_{aijk} \leq x ].
	\]
	El segundo test sí hace uso de los intervalos, notando 
por $[q_{-adf}, q_{+adf}]$ el intervalo de los errores para 
un algoritmo, base de datos y \textit{fold}. Se definen las 
cdf muestrales como
	
\begin{align*}
	\hat{F}_{-a \cdot \cdot}(x) &=
		\frac{1}{n_d n_f}
			\sum\limits_{i=1}^{n_d}
				\sum\limits_{j=1}^{n_f}
					\mathbb{I}[ q_{+aij} \leq x ], \\
	\hat{F}_{+a \cdot \cdot}(x) &=
		\frac{1}{n_d n_f}
			\sum\limits_{i=1}^{n_d}
				\sum\limits_{j=1}^{n_f}
					\mathbb{I}[ x \in [q_{-aij},q_{+aij}) ] +
			\hat{F}_{-a \cdot \cdot}(x) .			
\end{align*}

	En este test se aplica también el algoritmo 
~\ref{alg:CMC-pvalue}, rechazando la hipótesis nula si el 
intervalo $[\hat{q}_{-a}, \hat{q}_{+a}]$, con $\hat{q}_{-a}, 
\hat{q}_{+a}$ las medias de los extremos inferiores y 
superiores respectivamente para cada algoritmo con respecto a 
las conjuntos de datos y \textit{fold}, se encuentra en las colas 
de la distribución de los $[\hat{q}_{-a}^*, \hat{q}_{+a}^*]$.
\\
	Los resultados incluidos ofrecen una comparación entre la 
potencia de estos test presentados, el test de Friedman y el 
de ANOVA. Se realiza un experimento con datos sintéticos 
consistentes en los resultados de los errores de 5 algoritmos 
en 32 conjuntos de datos para una secuencia de diferencias en sus 
medias preestablecidas. Para calcular la potencia se estima 
para cada diferencia el porcentaje de experimentos para el 
que se encuentran diferencias significativas entre los 
algoritmos dos a dos. Los resultados indican una mayor 
potencia de los test presentados en este artículo.


\section{Revisión del estado del arte en la aplicación de test bayesianos}

	Se presenta en este apartado una revisión del estado del 
arte con diferentes métodos para la comparación de algoritmos
de aprendizaje automático mediante la realización de test 
estadísticos. 

\paragraph{Versión bayesiana del test de Wilcoxon} 
\cite{DBLP:conf/icml/BenavoliCMZR14} Para la comparación de algoritmos mediante
muestras apareadas, se presenta el test de Wilcoxon en 
versión bayesiana. Se realiza, al igual que en el test
de signo y el de rangos con signo, utilizando el Proceso de
Dirichlet para la distribución \textit{a priori}. La versión 
bayesiana, a diferencia de la no paramétrica, no necesita 
suponer la simetría de las muestras. Una cuestión importante
es la elección de $G_0$ en caso de que no dispongamos
información. El enfoque realizado se basa en la obtención
del límite de $DP$ para $s \rightarrow 0$. Otro enfoque 
consiste en modelar la falta de información \textit{a priori} 
como un conjunto de distribuciones de probabilidad. 

\paragraph{Procedimiento no paramétrico bayesiano para la 
comparación de algoritmos} \cite{DBLP:conf/icml/BenavoliCMZ15} En este 
artículo se presenta la versión bayesiana del test de 
Friedman. Se presenta el test, que utiliza el $DP$ para 
realizar la comparación de dos algoritmos en un primer
momento y la versión para tres o más. Una vez se ha rechazado 
la hipótesis de que los algoritmos se comporten de igual
manera, se procede realizando la comparación dos a dos 
para encontrar dónde se producen las diferencias. Se realiza 
también en este artículo un experimento comparando el test
bayesiano de Friedman con el test $F$-\textit{race}, test que
compara varios algoritmos según aquel que consiga el mejor
resultado.

\paragraph{Test estadístico para el análisis conjunto de 
medidas de rendimiento} \cite{DBLP:conf/jsai/BenavoliC15}	Para la 
comparación de algoritmos con respecto a varias medidas
se suelen utilizar o bien la media ponderada de las medidas
para cada algoritmo o bien el frente de Pareto, donde se 
consideran las soluciones no dominadas (aquellas que no son
peor con respecto a otra solución para todos los criterios).
En este artículo se presenta en primer lugar la realización 
de un test basado en la razón de verosimilitud, que presenta
las mismas desventajas que los THN. Para la realización del
test bayesiano, se opta por usar como distribución 
\textit{a priori} la distribución de Dirichlet, obteniéndose
también como distribución \textit{a posteriori} la 
distribución de Dirichlet de la que se calculan las 
probabilidades deseadas mediante muestras aleatorias.
Finalmente, se presenta una mejora de este método usando
redes bayesianas.

\paragraph{Comparación de algoritmos en múltiples bases de 
datos con CV} \cite{DBLP:journals/ml/CoraniB15} En este artículo se introduce
el $t$-test bayesiano (\ref{ssec:bayes-ttest}) y un test de 
Poisson que compara dos clasificadores en varias bases de 
datos teniendo en cuenta la correlación y la incertidumbre 
provocada por la validación cruzada en cada conjunto de 
datos.

\paragraph{Tutorial para la comparación de clasificadores 
mediante análisis bayesiano}. \cite{DBLP:journals/corr/BenavoliCDZ16} Este artículo 
es un artículo divulgativo que presenta las ventajas de los 
test bayesianos y expone su uso en diferentes casos, 
aplicando el $t$-test bayesiano, el test de signos y el $t$-
test bayesiano jerárquico, una extensión que permite realizar 
inferencia en varios conjuntos de datos.

\chapter{Software}

\chapter{Experimentos}
\section{Introducción al caso práctico}
	Una vez hemos presentado las principales técnicas
estadísticas que se utilizan en la comparación de algoritmos
en aprendizaje automático, realizaremos un estudio práctico
en \texttt{R} sobre la aplicación de estas técnicas.\\
	Para ello, usaremos los resultados obtenidos por cinco
algoritmos (Tabla~\ref{tab:algoritmos}) 
sobre varios conjuntos de datos (Tabla~\ref{tab:DS}). Los
conjuntos de datos constituyen un subconjunto de los 
disponibles en el repositorio 
\fnurl{KEEL}{http://www.keel.es/} 
(\cite{alcala2010keel}) para problemas de clasificación
supervisada. En dicho repositorio encontramos ya disponibles
particiones de los conjuntos de datos para facilitar que los
experimentos sean reproducibles. Se ha utilizado para cada 
conjunto de datos la partición para realizar $10CV$, esto es,
cada conjunto se particiona en 10 subconjuntos y en cada una 
de las 10 repeticiones, utilizamos un subconjunto como test y
los restantes 9 como datos de entrenamiento. Cada archivo
contiene los diez pares de ficheros con datos de
entrenamiento y test.

\begin{table}[H]
\centering
\caption{Algoritmos}
\label{tab:algoritmos}
\begin{tabularx}{\textwidth}{lX}
\toprule
Algoritmo             & Descripción                                                                                                                             \\ \midrule
\texttt{multinom}     & Regresión logística de la biblioteca \texttt{nnet} para un problema de clasificación con varias clases.                               \\
\texttt{knn}          & Algoritmo de los $k$ vecinos más cercanos de la biblioteca \texttt{class}. 
Se han utilizado los valores por defecto $k=1$, $l=0$.      \\
\texttt{randomForest} & Algoritmo de la biblioteca \texttt{randomForest}. Se ha fijado el parámetro \texttt{mtray} al valor por defecto $mtray = \sqrt{p}$. \\
\texttt{nnet}         & Red neuronal de la biblioteca \texttt{nnet}.                                                                                          \\
\texttt{naiveBayes}   & Clasificador \textit{naive Bayes} de la biblioteca \texttt{e1071}.                                                                  \\ \bottomrule
\end{tabularx}
\end{table}

\begin{table}[H]
\centering
\caption{Conjuntos de datos}
\label{tab:DS}
\begin{tabular}{@{}lccc@{}}
\toprule
Conjunto de datos & Número de atributos & Número de clases & Número de instancias \\ \midrule
abalone           & 8                   & 28               & 4174                 \\
australian        & 14                  & 2                & 690                  \\
automobile        & 25                  & 6                & 205                  \\
balance           & 4                   & 3                & 625                  \\
breast            & 9                   & 2                & 277                  \\
bupa              & 6                   & 2                & 345                  \\
car               & 6                   & 4                & 1728                 \\
cleveland         & 13                  & 5                & 303                  \\
crx               & 15                  & 2                & 690                  \\
dermatology       & 34                  & 6                & 366                  \\
german            & 20                  & 2                & 1000                 \\
glass             & 9                   & 7                & 214                  \\
hayes-roth        & 4                   & 3                & 160                  \\
heart             & 13                  & 2                & 270                  \\
ionosphere        & 33                  & 2                & 351                  \\
led7digit         & 7                   & 10               & 500                  \\
letter            & 16                  & 26               & 20000                \\
lymphography      & 18                  & 4                & 148                  \\
mushroom          & 22                  & 2                & 8124                 \\
optdigits         & 64                  & 10               & 5620                 \\
satimage          & 36                  & 7                & 6435                 \\
spambase          & 57                  & 2                & 4597                 \\
splice            & 60                  & 3                & 3190                 \\
tic-tac-toe       & 9                   & 2                & 958                  \\
vehicle           & 18                  & 4                & 846                  \\
vowel             & 13                  & 11               & 990                  \\
wine              & 13                  & 3                & 178                  \\
yeast             & 8                   & 10               & 1484                 \\
zoo               & 16                  & 7                & 101                  \\ \bottomrule
\end{tabular}
\end{table}
