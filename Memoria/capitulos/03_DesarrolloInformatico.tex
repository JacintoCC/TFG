%\documentclass[11pt,leqno]{book}
%\usepackage[spanish,activeacute]{babel}
%\usepackage[utf8]{inputenc}
%\usepackage{enumerate}
%
%\begin{document}

\chapter{Desarrollo informático}

\section{Aplicación de test estadísticos en el aprendizaje automático}

\subsection{Introducción}

	A la hora de evaluar un algoritmo de aprendizaje 
automático nos interesa conocer el error para un problema y 
su relación con respecto a otros, sin embargo, debemos ser 
conscientes de que para ello debemos diseñar y realizar un 
experimento cuyos resultados sean fiables y minimizar el 
efecto producido por las circunstancias en las que se 
realizan o los datos disponibles. Por ejemplo, el error 
obtenido en los datos utilizados para entrenar el modelo 
(error de entrenamiento) será menor que el error en unos 
datos que no han participado en este entrenamiento y son 
usados para evaluar el modelo (error de test). Cabe mencionar 
también que el error de entrenamiento no es comparable debido 
a que modelos más complejos se ajustarán más a los datos de 
entrenamiento, sin que eso signifique que el modelo se 
comporte mejor de manera global. Hay algoritmos que se 
comportan de manera no determinista, y por tanto deseemos 
realizar varias ejecuciones para estimar circunstancias 
aleatorias. En definitiva, debemos basar nuestra evaluación 
del algoritmo en la distribución (desconocida) del error. 
Para una correcta estimación del error es importante tener en 
cuenta lo siguiente:
	
	\begin{itemize}
	\item El error depende del problema. El teorema de 
		\textit{No Free Lunch} afirma que dados dos 
		algoritmos su rendimiento medio para todos los 
		posibles problemas es el mismo, lo que se traduce en 
		que según el problema se puede dar que un algoritmo 
		sea mejor que otro o al contrario.
	\item La partición de la base de datos referente al 
		problema en datos de entrenamiento y de test se 
		realiza para obtener una idea más acertada del error 
		cometido por el algoritmo para el problema. Sin 
		embargo, una vez que se considere válido el algoritmo 
		se entrenará con todos los datos disponibles.
	\item Para validar el algoritmo, se deben utilizar datos 
		que no hayan sido utilizados anteriormente de ninguna 
		manera.
	\item Normalmente, para la comparación del rendimiento de 
		algoritmos se usa el error cometido en la 
		clasificación, aunque se podrían considerar otros 
		factores como el tiempo necesario para entrenar el 
		algoritmo, su simplicidad...
	\end{itemize}
	
\subsubsection{Diseño de experimentos}

	Al hablar de experimentos nos referimos al test o a la 
serie de test donde se manejan factores que afectan a la 
salida. Pretendemos minimizar aquellos factores que no se 
controlan y obtener resultados estadísticamente 
significativos. Para ajustar parámetros propios del modelo de 
aprendizaje automático que estemos utilizando existen 
diferentes aproximaciones: la \textit{mejor suposición} 
(donde quien realiza el experimento ajusta los parámetros 
basándose en su conocimiento sobre el problema), \textit{un 
factor cada vez} (se supone que los factores son 
independientes y se prueban diferentes valores para cada 
parámetro de uno en uno, mientras los demás se mantienen en 
una posición base) y \textit{diseño factorial} (se crea una 
rejilla con diferentes posibles valores para cada parámetro y 
se prueba en ellos, es más costoso).\\

	Para minimizar el número de ejecuciones necesarias, se 
suele utilizar conocimiento previo para reducir las 
combinaciones a probar. Otra estrategia consiste en el diseño 
de la superficie de respuesta. Se considera la siguiente 
situación:
		\[ r = g( f_1, \dots, f_F | \phi ), \]
	donde $r$ es la respuesta, $g$ el modelo utilizado, 
$f_1, \dots f_F$ los factores y $\phi$ la estimación empírica 
del modelo para cada configuración particular probada. El 
procedimiento consiste en ir añadiendo a $\phi$ las 
evaluaciones realizadas, ajustar $g$ según $\phi$ y buscar su 
máximo (o mínimo), evaluar el algoritmo en ese punto y 
ajustar $\phi$.\\
	En la realización de experimentos es necesaria la 
aleatorización a la hora de probar las diferentes 
configuraciones de parámetros (no tanto en nuestro contexto, 
pero el orden de las configuraciones al realizar experimentos 
donde interviene maquinaria por ejemplo podría inferir en los 
resultados debido a factores como la temperatura de la 
máquina). Otro factor a tener en cuenta es que es necesaria 
la repetición de las ejecuciones y promediar los resultados 
para así reducir el impacto de factores aleatorios. Para 
reducir la variabilidad de factores que no dependen del 
propio algoritmo a evaluar, como las bases de datos o las 
particiones realizadas, en la comparación de varios 
algoritmos utilizamos las mismas particiones para cada uno.

\paragraph{Directrices a la hora de realizar un experimento}
	\begin{enumerate}
	\item Fijar la intención del estudio
	\item Seleccionar las variables de respuesta
	\item Seleccionar los factores con los que se realizará 
		el experimento y los niveles a comprobar
	\item Diseño del experimento: La partición para 
		entrenamiento y test depende del tamaño de la base de 
		datos. Si la base de datos es pequeña puede haber una 
		alta variabilidad y obtener resultados no 
		concluyentes.
	\item Efectuar el experimento: Antes de llevar a cabo un 
		experimento de gran magnitud es aconsejable probar 
		pequeños conjuntos de datos para probar que funciona. 
		También es interesante guardar resultados intermedios 
		o las semillas aleatorias para facilitar la 
		reproducción de los datos.
	\item El experimentador debe ser imparcial y juzgar de 
		igual manera un algoritmo que otro, realizar 
		documentación ...
	\item Análisis estadístico: las afirmaciones y preguntas 
		que se realicen debe sostenerse estadísticamente 
		hablando. Es aconsejable un análisis visual para 
		exponer los datos obtenidos.
	\end{enumerate}
	
	
\subsubsection{Validación cruzada y remuestreo}
	
	Para la repetición de los experimentos, necesitamos un 
número de conjuntos de entrenamiento y validación del 
conjunto de datos disponible. Si el conjunto es 
suficientemente grande, podemos dividir aleatoriamente el 
conjunto en $K$ partes, y de cada parte seleccionar la mitad 
para entrenamiento y la otra mitad para test. $K$ suele ser 
10 ó 30. Sin embargo, los conjuntos de datos no suelen ser 
tan grandes. Por tanto, lo que habitualmente se realiza es 
utilizar varias veces los mismos datos pero de formas 
distintas, lo que se conoce como \textbf{validación cruzada} 
(\textit{cross-validation}, CV). Otro concepto a tener en 
cuenta es el de \textbf{estratificación}. Consiste en 
mantener la proporción de las diferentes clases en cada una 
de las particiones realizadas en la base de datos. 
	
\paragraph{Validación cruzada con $K$-fold} Se divide el 
conjunto de datos $\mathcal{X}$ en $K$ partes aleatoriamente, 
$\mathcal{X} = \mathcal{X}_1 \cup \dots \cup \mathcal{X}_K$. 
Para formar cada par de datos de entrenamiento y validación, 
$(\mathcal{T}, \mathcal{V})$, se mantiene una de las $K$ 
partes y se unen las demás partes como conjunto de 
entrenamiento
\begin{align*}
	\mathcal{V}_i &= \mathcal{X}_i \\
	\mathcal{T}_i &= \underset{j=1,\dots,K; j\neq i}
							\bigcup \mathcal{X}_j
\end{align*} 
	El problema de este método es que el conjunto de 
validación es pequeño (lo que lleva a una mayor variabilidad 
en el error). Además, los conjuntos de entrenamiento se 
solapan considerablemente. $K$ es habitualmente 10 ó 30. 
Conforme $K$ crece, se incrementa la robustez al afectar 
menos cada dato concreto. El caso $K = n-1$, se conoce como 
\textit{leave-one-out}. 

\paragraph{$5 \times 2$ CV} Otra propuesta consiste en 
dividir $\mathcal{X}$ en dos partes, utilizar una de estas 
partes como datos de entrenamiento y la otra como test e 
invertir los roles. Para obtener el segundo \textit{fold}, se 
realiza otra partición aleatoria de $\mathcal{X}$ y se repite 
el proceso. Lo habitual es que se realice cinco veces, 
obteniendo así diez parejas $(\mathcal{T}, \mathcal{V})$. 
Aunque podría realizarse más veces, al realizarse más veces 
los conjuntos comparten muchas instancias y por lo tanto los 
errores son muy dependientes entre sí, con lo que no se 
aporta nueva información. Si se realiza menos de cinco veces, 
se obtienen pocas muestras y es más complejo comprobar las 
hipótesis.

\paragraph{\textit{Bootstrapping}} Para generar varias 
muestras a partir de una única se seleccionan instancias de 
la original con reemplazamiento. Puede solaparse más que con 
CV, con lo que las estimaciones resultan más dependientes. 
Por ello, se recomienda especialmente para conjuntos de datos 
muy pequeños en los que se necesiten mayores muestras. 

	Al realizar \textit{bootstrap}, se seleccionan $N$ 
instancias de un conjunto de tamaño $N$ con reemplazamiento. 
El conjunto original se usa como conjunto de validación. La 
probabilidad de no seleccionar una instancia es $1-1/N$, con 
lo que la probabilidad de no seleccionarla tras $N$ 
elecciones es $\left(1- \frac{1}{N}\right)^N \approx e^{-1} 
\approx 0.368$. Esto significa que el conjunto de 
entrenamiento contiene aproximadamente un $63.2\%$ de los 
datos. Para obtener una mejor estimación del error, se 
propone repetir el proceso y observar el comportamiento 
medio.
	
\subsubsection{Test de hipótesis}

	En la realización de los test de hipótesis, el 
procedimiento es el siguiente: se define un estadístico que 
sigue una distribución conocida en el caso de que se cumpla 
la hipótesis realizada. Entonces si el estadístico calculado 
de la muestra tiene poca probabilidad de obtenerse de la 
distribución, se rechaza la hipótesis; en caso contrario, no. 
\\
	Surgen entonces las cuestiones: ¿Se puede justificar los 
resultados estadísticamente o se obtienen por azar? ¿Son los 
conjuntos de datos representativos para el problema? Estas 
preguntas no se pueden responder de forma exhaustiva, es la 
labor de los test estadísticos reunir los datos disponibles 
para justificar la consistencia de las conclusiones. \\
	A la hora de realizar los test estadísticos podemos 
realizar las siguientes comparaciones:
	\begin{enumerate}
	\item Comparación de dos algoritmos en un dominio: $t$-
		test (paramétrico), test de McNemar (no paramétrico).
	\item Comparación de múltiples algoritmos en un dominio: 
		\textit{Signed-Rank test} de Wilcoxon y test de signo 
		(no paramétricos).
	\item Comparación de múltiples algoritmos en varios 
		dominios: ANOVA (paramétrico), test de Friedman o 
		basados en permutaciones (no paramétricos). 
	\end{enumerate}

	Nótese que el test de \textit{Signed-Rank} de Wilcoxon 
para muestras emparejadas puede aplicarse tanto para la 
comparación de dos como entre varios algoritmos en un 
dominio. Los test para varios algoritmos en varias bases de 
datos necesitan un test posterior en caso de rechazar la 
hipótesis de que todos los algoritmos se comporten de igual 
manera. Ejemplos de estos test son el de Tukey, de 
Bonferroni-Dunn o de Nemenyi. 
	
	
	
	
	
	
	
	
%\end{document}
