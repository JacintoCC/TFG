%\documentclass[11pt,leqno]{book}
%\usepackage[spanish,activeacute]{babel}
%\usepackage[utf8]{inputenc}
%\usepackage{enumerate}
%
%\begin{document}


\chapter{Desarrollo matemático}

\section{Conceptos básicos de la inferencia}

	En esta primera sección se hace un breve repaso de los conceptos estadísticos necesarios para comprender el contenido de la memoria así como presentar la notación.
	
\begin{definicion}[Inferencia estadística]
	Rama de la estadística en la que se usan las propiedades de una muestra para extraer conclusiones de la población. 
\end{definicion}

\begin{definicion}[Espacio muestral] 
	Conjunto de los posibles resultados de un experimento aleatorio
\end{definicion}

\begin{definicion}[Variable aleatoria]
	Función de conjuntos cuyo dominio es los elementos de un espacio muestral sobre el cual se ha definido una función de probabilidad y cuyo rango es $\mathbb{R}$.\\
	$X$ es una variable aleatoria (v.a.) si para $x \in \mathbb{R}$ existe una probabilidad de que el valor tomado por la variable aleatoria sea menor o igual que $x$, es decir, $P(X \leq x) = F_X (x)$, llamada función de distribución de probabilidad (\textit{cumulative distribution function}, cdf) de $X$.	
\end{definicion}

Cualquier función de distribución, $F_X(x)$, de una v.a. $X$ cumple las siguientes propiedades: 
\begin{enumerate}
	\item $F_X$ es no decreciente: 
			$F_X(x_1) \leq F_X(x_2) \forall x_1 \leq x_2$.
	\item $\lim_{x \rightarrow -\infty} F_X(x) = 0$,
			$\lim_{x \rightarrow \infty} F_X(x) = 1$
	\item $F_X(x)$ es continua por la derecha: 
		$\lim_{\varepsilon \rightarrow 0^+} F_X(x+\varepsilon) = F_X(x)$
\end{enumerate}

	Diremos que una v.a. es \textbf{continua} si su cdf es continua. Supondremos que una cdf continua es derivable cpd (\textit{casi por doquier}, es decir, en todo $\mathbb{R}$ salvo en un conjunto finito de puntos).
	
\begin{definicion}[Función de densidad]
	Se define la función de densidad como la derivada de $F_X(x)$, $f_X(x)$. Para $X$ continua:
	\[ F_X(x) = \int_{-\infty}^x f_X(t) dt \quad
		f_X(x) = \frac{d}{dx}F_X(x) = F_X'(x) \geq 0 \quad
		\int_{-\infty}^{\infty} f_X(x) dx = 1 \]
\end{definicion}
	
\begin{definicion}[Función de masa]
	Se define la función de masa de probabilidad  (\textit{probability mass function}, pmf) de una v.a. \textbf{discreta}, es decir, que sólo toma un número contable de valores como
	\[ 
	f_X(x) = P(X=x) = 
		F_X(x) - 
		\lim_{\varepsilon \rightarrow 0^+} F_X(X-\varepsilon)
	\]
\end{definicion}

	Usaremos el término función de probabilidad (pf) para referirnos a una pdf o una pmf indistintamente.
	
\begin{definicion}[Esperanza matemática]
	Se define la esperanza matemática de una función $g(X)$ o una v.a. $X$, $E[g(X)]$ como
	\[ E[g(X)] = \left\lbrace 
		\begin{array}{cc}
		\int_{-\infty}^{\infty} g(x)f_X(x) dx &
			\textit{si } X \textit{ es continua} \\
		\sum\limits_{\textit{todo } x} g(x)f_X(x) dx &
			\textit{si } X \textit{ es discreta} \\			
		\end{array}\right.
	\]
\end{definicion}


	
\section{Test paramétricos}

\section{Test no paramétricos}

	\subsection{Test basados en permutaciones}

\section{Test bayesianos}	
%\end{document}