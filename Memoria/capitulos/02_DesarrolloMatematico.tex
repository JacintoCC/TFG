%\documentclass[11pt,leqno]{book}
%\usepackage[spanish,activeacute]{babel}
%\usepackage[utf8]{inputenc}
%\usepackage{enumerate}
%
%\begin{document}


\chapter{Desarrollo matemático}

\section{Introducción a la inferencia}
	
	En esta primera sección se hace un breve repaso de los conceptos estadísticos necesarios para comprender el contenido de la memoria así como presentar la notación.
	
\begin{definicion}[Inferencia estadística]
	Rama de la estadística en la que se usan las propiedades de una muestra para extraer conclusiones de la población. 
\end{definicion}

\begin{definicion}[Espacio muestral] 
	Conjunto de los posibles resultados de un experimento aleatorio
\end{definicion}

\begin{definicion}[Variable aleatoria]
	Función de conjuntos cuyo dominio es los elementos de un espacio muestral sobre el cual se ha definido una función de probabilidad y cuyo rango es $\mathbb{R}$.\\
	$X$ es una variable aleatoria (v.a.) si para $x \in \mathbb{R}$ existe una probabilidad de que el valor tomado por la variable aleatoria sea menor o igual que $x$, es decir, $P(X \leq x) = F_X (x)$, llamada función de distribución de probabilidad (\textit{cumulative distribution function}, cdf) de $X$.	
\end{definicion}

Cualquier función de distribución, $F_X(x)$, de una v.a. $X$ cumple las siguientes propiedades: 
\begin{enumerate}
	\item $F_X$ es no decreciente: 
			$F_X(x_1) \leq F_X(x_2) \forall x_1 \leq x_2$.
	\item $\lim_{x \rightarrow -\infty} F_X(x) = 0$,
			$\lim_{x \rightarrow \infty} F_X(x) = 1$
	\item $F_X(x)$ es continua por la derecha: 
		$\lim_{\varepsilon \rightarrow 0^+} F_X(x+\varepsilon) = F_X(x)$
\end{enumerate}

	Diremos que una v.a. es \textbf{continua} si su cdf es continua. Supondremos que una cdf continua es derivable cpd (\textit{casi por doquier}, es decir, en todo $\mathbb{R}$ salvo en un conjunto finito de puntos).
	
\begin{definicion}[Función de densidad]
	Se define la función de densidad como la derivada de $F_X(x)$, $f_X(x)$. Para $X$ continua:
	\[ F_X(x) = \int_{-\infty}^x f_X(t) dt \;
		f_X(x) = \frac{d}{dx}F_X(x) = F_X'(x) \geq 0 \;
		\int_{-\infty}^{\infty} f_X(x) dx = 1 \]
\end{definicion}
	
\begin{definicion}[Función de masa]
	Se define la función de masa de probabilidad  (\textit{probability mass function}, pmf) de una v.a. \textbf{discreta}, es decir, que sólo toma un número contable de valores como
	\[ 
	f_X(x) = P(X=x) = 
		F_X(x) - 
		\lim_{\varepsilon \rightarrow 0^+} F_X(X-\varepsilon)
	\]
\end{definicion}

	Usaremos el término función de probabilidad (pf) para referirnos a una pdf o una pmf indistintamente.
	
\begin{definicion}[Esperanza matemática]
	Se define la esperanza matemática de una función $g(X)$ o una variable aleatoria $X$, $E[g(X)]$ como
	\[ E[g(X)] = \left\lbrace 
		\begin{array}{cc}
		\int_{-\infty}^{\infty} g(x)f_X(x) dx &
			\textit{si } X \textit{ es continua} \\
		\sum\limits_{x} g(x)f_X(x) dx &
			\textit{si } X \textit{ es discreta}	
		\end{array}\right.
	\]
\end{definicion}

\begin{definicion}[Independencia]
	Un conjunto de $n$ v.a. se dice independiente sii su función de probabilidad conjunta es igual al producto de las $n$ funciones de probabilidad marginales.
\end{definicion}

\begin{definicion}[Muestra aleatoria]
	Llamamos muestra aleatoria de una v.a. $X$ a un conjunto de $n$ v.a., $(X_1, \dots, X_n)$, si son independientes e idénticamente distribuidas (i.i.d.), con lo que su distribución de probabilidad conjunta es
	\[ 
		f_{X}(x_1, \dots, x_n) =
		f_{X_1, \dots, X_n}(x_1, \dots, x_n) =
		\prod\limits_{i=1}^n f_X(x_i)
	\]
\end{definicion}
	
\begin{definicion}[Momento]
	Es un parámetro de la población. El momento $k$-ésimo de $X$ es $\mu_k' = E[X^k]$. La media es el momento de primer orden, $\mu_1' = E[X] = \mu$. El momento $k$-ésimo central es $\mu_k = E[(X - \mu)^k]$
\end{definicion}

\begin{definicion}[Covarianza]
	La covarianza entre dos v.a. $X,Y$, se define como
		\[	cov(X,Y) = E[(X-\mu_X) (Y-\mu_Y)] = 
						E[XY] - \mu_X \mu_Y 
		\]
\end{definicion}

\begin{definicion}[Función generadora de momentos]
	La función generadora de momentos (\textit{moment-generating function}, mgf) de una función $g(X)$ de $X$ es
	\[ M_{g(X)}(t) = E[ \exp(tg(X)) ] \]
\end{definicion}

\begin{teorema}[Desigualdad de Chebyshev]
	Sea $X$ una v.a. con media $\mu$ y varianza $\sigma^2 < \infty$. Entonces para $k > 0$ se cumple
		\[ P(|X - \mu| \geq k) \leq \frac{\sigma^2}{k^2} \]
\end{teorema}

\begin{teorema}[Teorema central del límite]
	Sea $X_1, \dots, X_n$ una muestra aleatoria de una población con media $\mu$ y varianza $\sigma^2 > 0$ y sea $\bar{X}_n$ la media de esa muestra. Entonces para $n \rightarrow \infty$ la variable aleatoria $\sqrt{n} \frac{(\bar{X}_n - \mu)}{\sigma}$ tiene como distribución límite la normal con media $0$ y varianza $1$.
\end{teorema}

\begin{definicion}[Estimador]
	Definimos como estimador, o estimador puntual una función de v.a. cuyo valor observado es usado para estimar el valor verdadero de un parámetro de la población. 
\end{definicion}

	Sea $\hat{theta}_n = u(X_1, \dots, X_n)$ un estimador de un parámetro $\theta$. Incluimos unas propiedades deseables de $\hat{theta}_n$:
	
	\begin{enumerate}
	\item \textit{Insesgadez}: 
			$E[\hat{\theta}_n = \theta$ para todo $\theta$.
	\item \textit{Suficiencia}: Podemos escribir
			$f_{X_1, \dots, X_n}(x_1, \dots, x_n; \theta)$ como producto de dos funciones $f_{X_1, \dots, X_n}(x_1, \dots, x_n; \theta) = g(\hat{\theta}_n; \theta) H((x_1, \dots, x_n)$ tal que $H(x_1, \dots, x_n)$ no depende de $\theta$.
	\item \textit{Consistencia}
		\[ \lim_{n \rightarrow \infty} P(|\bar{\theta}_n - \theta| < \varepsilon) = 0 \quad \forall \varepsilon > 0 \]
	\begin{enumerate}
		\item Si $\hat{\theta}_n$ es un estimador insesgado de $\theta$ y $\lim_{n \rightarrow \infty} var(\hat{\theta}_n) = 0$, entonces $\hat{\theta}_n$ es un estimador consistente por la desigualdad de Chebyshev.
		\item $\hat{\theta}_n$ es un estimador consistente de $\theta$ si la distribución límite es la distribución degenerada con probabilidad $1$ en $\theta$.
	\end{enumerate}
			
	\item \textit{Mínimo error cuadrático} 	$E[(\hat{\theta}_n - \theta)^2] \leq E[(\hat{\theta}_n^\star - \theta)^2]$ para cualquier estimador $\hat{\theta}_n^\star$.
	\item \textit{Mínima varianza} 	$var(\hat{\theta}_n) \leq var(\hat{\theta}^\star_n)$ para cualquier estimador $\hat{\theta}_n^\star$, siendo ambos insesgados.
	\end{enumerate}
	
\begin{definicion}[Función de verosimilitud]
	La función de verosimilitud (\textit{likelihood function}) de una muestra aleatoria de tamaño $n$ de la población $f_X(x;\theta)$ es la probabilidad conjunta de las muestras tomadas como función de $\theta$. Esto es:
	\[ L(x_1, \dots, x_n; \theta) = 
		\prod\limits_{i=1}^n f_X(x_i;\theta)	\]
\end{definicion}

	Un \textbf{estimador máximo verosímil} (MLE) de $\theta$ es un valor $\bar{\theta}$ tal que 
	\[ L(x_1, \dots, x_n; \bar{\theta}) \geq 
			L(x_1, \dots, x_n; \theta) \forall \theta 	\]
	La relevancia de este estimador consiste en que, para unas ciertas condiciones de regularidad, un estimador máximo verosímil es suficiente, consistente y asintóticamente insesgado, con varianza mínima y con distribución normal.
	
	
\begin{definicion}[Intervalo de confianza]
	Un intervalo de confianza al $100(1-\alpha)\%$ para el parámetro $\theta$ es un intervalo aleatorio de extremos $U$ y $V$ (funciones de v.a.) tal que $P(U < \theta < V) = 1-\alpha$.
\end{definicion}
	
\begin{definicion}[Hipótesis estadística]
	Es una afirmación sobre la la función de probabilidad de una o más v.a. o una afirmación sobre las poblaciones de las cuales se han obtenido una o más muestras aleatorias. La \textbf{hipótesis nula}, $H_0$ es la hipótesis sobre la que se realizará un test. La \textbf{hipótesis alternativa}, $H_1$ es la que conclusión alcanzada si se rechaza la hipótesis nula.
\end{definicion}

\begin{definicion}[Región crítica]
	Llamamos región crítica o región de rechazo $R$ para un test al conjunto de valores tomados por el test que conducen a rechazar la hipótesis nula. Llamamos \textbf{valores críticos} a los extremos de $R$.
\end{definicion}

\begin{definicion}[Tipos de error]\textit{}
	\begin{description}
	\item[Error de tipo I] La hipótesis nula es rechazada siendo cierta.
	\item[Error de tipo II] La hipótesis nula no es rechazada siendo falsa.
	\end{description}
\end{definicion}

	Siendo $T$ un test estadístico con hipótesis $H_0: \theta \in \omega, \ H_1: \theta \in \Omega \setminus \omega$, los errores de tipo I y II tienen probabilidad
	\[ 
	\alpha(\theta) = P(T \in R | \theta \in \omega); \quad
	\beta(\theta) = 
		P(T \not\in R | 
				\theta \in \Omega \setminus \omega)
	\]
	respectivamente.

\begin{definicion}[Tamaño del test]
	Se define el tamaño del test como $\sup_{\theta \in \omega} \alpha(\theta)$.
\end{definicion}

\begin{definicion}[Potencia del test]
	Se define la potencia del test como la probabilidad de que el test conduzca a un rechazo de $H_0$: $Pw(\theta) = P(T \in R)$. Esta medida nos interesa cuando debemos rechazar la hipótesis nula, con lo que calculamos $Pw(\theta) = P(T \in R | \theta \in \Omega \setminus \omega) = 1 - \beta(\theta)$. 
\end{definicion}
	
	Diremos que un test es \textbf{más potente} para una hipótesis alternativa concreta si ningún test del mismo tamaño tiene mayor potencia contra la misma hipótesis alternativa.\\
	A continuación definimos una aproximación alternativa a la realización de test de hipótesis, especialmente relevante en los test no paramétricos. 
	
\begin{definicion}[$p$-valor]
	Probabilidad, siendo cierta la hipótesis nula $H_0$, de obtener una muestra aleatoria con un valor del parámetro $\theta$ sobre el que se realiza el test tanto o más alejado que el observado en una muestra. 
\end{definicion}

\begin{definicion}[Consistencia]
	Diremos que un test es consistente para una hipótesis alternativa $H_1$ si la potencia del test se aproxima a 1 conforme $n \rightarrow \infty$, siendo $n$ el tamaño de la muestra.
\end{definicion}
		
	
	
\section{Test paramétricos}

	 
	
	 

\section{Test no paramétricos}

	En la inferencia clásica se efectúan suposiciones sobre la población de la que se extraen muestras para realizar la inferencia. Aunque estas suposiciones están normalmente justificadas, en ocasiones no se dan las circunstancias necesarias para aplicar estas técnicas o su uso no está bien documentado. Por ello surgen las técnicas no paramétricas.
	
	\subsection{Comparación con test paramétricos}
		 
 	La principal ventaja de los test no paramétricos es que las hipótesis son más generales con lo que se pueden aplicar en un mayor número de problemas. Una de las condiciones habituales es la continuidad, aunque hay otras condiciones más estrictas como que la población sea simétrica para según qué test. Esto repercute en que se le pueden aplicar funciones a las muestras obtenidas para la realización de los test, a diferencia de en los test paramétricos dado que las muestras deben generalmente provenir de una población de forma conocida.\\
 	
 	Hay test de hipótesis que no están relacionados con valores de parámetros (a diferencia de los paramétricos). Son más simples de aplicar, las matemáticas, menos sofisticadas y basadas en la combinatoria, están relacionadas con las propiedades usadas en el proceso inductivo. Los libros de recetas no son necesarios, pues con la mera definición del test queda suficientemente claro cómo aplicar el test no paramétrico. Además, las distribuciones asintóticas son distribuciones conocidas como la normal o la chi cuadrado. Al relajar las condiciones sobre los datos de entrada, es menos sensible al \textit{dirty-data}, datos con errores usados en el entrenamiento del clasificador. Esto implica una mayor robustez en los test no paramétricos.\\
 	
 	Correctamente aplicados, los test paramétricos, al disponer de mayor información tienen una mayor potencia, sin embargo, cuando se disponen de menos datos, y por tanto es más difícil que se den las condiciones de los test paramétricos, la potencia es similar.

	\subsection{Desarrollo de los TNP}
		 
 	Se incluye en esta sección el desarrollo de los test no paramétricos más utilizados en aprendizaje automático.
 	
 	
\subsubsection{Test de aleatoriedad}

	Una de las condiciones para la realización de los test estadísticos, tanto de los paramétricos como de los no paramétricos, es la aleatoriedad de la muestra de partida. La hipótesis nula para los test que serán presentados en esta sección será la aleatoriedad de la muestra, mientras que la hipótesis alternativa será la presencia de un patrón. Estos test también son útiles en los estudios de series temporales y control de calidad.
	
\subsubsection*{Test basado en el número de rachas}

	Supongamos una secuencia de $n$ elementos de dos tipos, $n_1$ del primer tipo y $n_2$ del segundo, $n = n_1 + n_2$. Sea $R_1$ el número de rachas del primer tipo, $R_2$ el número de rachas del segundo tipo, $R_ = R_1 + R_2$. Siendo cierta la hipótesis nula (la aleatoriedad de la muestra), procedemos a obtener la distribución de $R$.
	
\begin{lema} 
	El número de formas distintas de distribuir $n$ objetos en $r$ posiciones consecutivas es ${n-1 \choose r-1}, n \geq r, r \geq 1$.
\end{lema}

\begin{teorema}
	Sean $R_1$ y $R_2$ los números de rachas de los $n_1$ de tipo 1 y los $n_2$ elementos de tipo 2 respectivamente en una muestra de tamaño $n = n_1 + n_2$. La función de distribución de probabilidad conjunta de $R_1$ y $R_2$ es
	\[ f_{R_1,R_2} (r_1, r_2) = 
		\frac{c {n_1 - 1 \choose r_1 - 1} 
				{n_2 - 1 \choose r_2 - 1}}
			{{n_1 + n_2 \choose n_1}}\;
		\begin{array}{l}
			r_1 = 1,2, \dots, n_1 \\
			r_2 = 1,2, \dots, n_2 \\
			r_1 = r_2 \textit{ or } r_1 = r_2 \pm 1
		\end{array}
	\]
	donde $c=2$ si $r_1 = r_2$ (hay igual número de rachas de elementos del tipo 1 y del tipo 2) y $c=1$ si $r_1 = r_2 \pm 1$ (hay una racha más del tipo 1 ó 2).
\end{teorema}

	Para muestras de un mayor tamaño (aquellas en el que $n_1, n_2 \geq 10$) se suele utilizar una aproximación utilizando la distribución asintótica supuesto cierta $H_0$.\\
	Suponemos que el tamaño de la muestra $n \rightarrow \infty$, de forma en que $\frac{n_1}{n} \rightarrow \lambda$, $0<\lambda<1$. De aquí obtenemos
	\[ \lim\limits_{n \rightarrow \infty} E[R/n] = 
			2\lambda (1-\lambda) 
				\lim\limits_{n \rightarrow \infty} 
					var(R\sqrt{n}) =
			4\lambda^2(1-\lambda)^2
	\]
	
\begin{teorema}
	La distribución de probabilidad de $R$, es decir, el número total de rachas en una muestra aleatoria es:
	
	\begin{equation}
		f_R(r) = \left\lbrace\begin{array}{ll}
	2 {n_1-1 \choose r/2-1} {n_2-1 \choose r/2-1} 
		\bigg/ {n_1 + n_2 \choose n_1} &
			\textit{ si } r \textit{ es par} \\
	\left[
		{n_1-1 \choose (r-1)/2} {n_2-1 \choose (r-3)/2} +  
		{n_1-1 \choose (r-3)/2} {n_2-1 \choose (r-1)/2} 
	\right]
		\bigg/ {n_1 + n_2 \choose n_1} &
			\textit{ si } r \textit{ es impar} \\		
		\end{array}\right.
	\end{equation}
	para $r=2, 3, \dots, n_1 + n_2.$
\end{teorema}
	
	Si llamamos $Z = \frac{R - 2n\lambda (1-\lambda)}{2 \sqrt{n}\lambda (1-\lambda)}$ y sustituimos en [], obtenemos la distribución estandarizada de $R$, $f_Z(z)$. Entonces aplicamos la fórmula de Stirling y el límite queda de la forma
	\[ \lim\limits_{n \rightarrow \infty} ln f_Z(z)=
			-ln \sqrt{2\pi} - \frac{1}{2} z^2	\]
	con lo que la distribución límite de $Z$ es la normal. 
	
	
\subsubsection*{Test basado en rachas crecientes y decrecientes}	

	Para este test consideramos una serie de datos de tipo numérico ordenados temporalmente y queremos comprobar la hipótesis de la aleatoriedad de la muestra.\\
	Para una muestra de $n$ elementos, supongamos que podemos ordenarlos de la forma $a_1 < \dots < a_n$ (estamos suponiendo que no hay dos iguales. Si la hipótesis nula fuese cierta, nuestra muestra se corresponderá con una de las $n!$ permutaciones con igual probabilidad. Usaremos para este test las rachas crecientes y decrecientes. Construimos la secuencia $D_{n-1}$, cuyo elemento $i$-ésimo es el signo de $x_{i+1} - x_i,\ i=1, \dots, n-1$. Sean $R_1, \dots, R_{n-1}$ el número de rachas de longitud $1, \dots, n-1$ respectivamente. $f_n(r_{n-1}, \dots, r_1)$ indica la probabilidad de obtener $r_j$ rachas de longitud $j$ supuesta cierta la hipótesis nula. Escribiremos como $u_n$ la frecuencia absoluta $f_n = \frac{u_n}{n!}$. Para obtener la función de distribución, partiremos del caso $n=3$\\
	Sean $a_1 < a_2 < a_3$. La distribución de probabilidad será 
	\[ 
	f_3(r_2, r_1) = 
		\left\lbrace\begin{array}{cc}
			\frac{2}{6} & \text{si } r_2 = 1, r_1 = 0 \\
			\frac{4}{6} & \text{si } r_2 = 0, r_1 = 2 
		\end{array}\right.
	\]
	dado que las únicas rachas de longitud 2 son $(a_1, a_2, a_3) \rightarrow (+,+)$ y $(a_3, a_2, a_1) \rightarrow (-,-)$, siendo las demás posibles combinaciones dos rachas de longitud 1.\\
	Las posibilidades a la hora de insertar un elemento $a_n$ en las permutaciones de $S_n$ son:
\begin{enumerate}
	\item Se añade una racha de longitud 1.
	\item Una racha de longitud $i-1$ se convierte en una de longitud $i$, $i=2,\dots n-1$.
	\item Una racha de longitud $h=2i$ se convierte en una de longitud $i$, seguida por otra de longitud $1$, seguida por otra de longitud $i$.
	\item Una racha de longitud $h=i+j$ se convierte en
	\begin{enumerate}
		\item Una racha de longitud $i$ seguida por otra de longitud $1$ seguida por otra de longitud $j$.
		\item Una racha de longitud $j$ seguida por otra de longitud $1$ seguida por otra de longitud $i$.
	\end{enumerate}
	con $h>i>j$, $3 \leq h \leq n-2$
\end{enumerate}

	De forma general, la frecuencia $u_n$ conocido $u_{n-1}$ sigue la siguiente relación:
\begin{align}
	& u_n (r_{n-1}, \dots, r_h, \dots, r_i, \dots, r_j, \dots, r_1)= 
		2 u_{n-1}(r_{n-2}, \dots, r_1-1) \\
	&+ \sum\limits_{i=2}^{n-1} 
		(r_{i-1} + 1)
		u_{n-1}(r_{n-2},\dots, r_i-1, r_{i-1}+1,\dots, r_1)\\
	&+ \sum\limits_{i=1, h=2i}^{\lfloor (n-2)/2 \rfloor} 
		(r_{h} + 1)
		u_{n-1}(r_{n-2},\dots, r_h+1,\dots r_i-2,\dots, r_1-1)\\
	&+ 2 \underbrace{\sum\limits_{i=2}^{n-3} \sum\limits_{j=1}^{i-1}}_{h=i+j, h \leq n-2}
		(r_{h} + 1)
		u_{n-1}(r_{n-2},\dots, r_h+1,\dots, r_i-1,\dots, r_1-1)			
\end{align}
	
	Otro test que se podría realizar es el del número total de rachas, independientemente de su longitud. El número de rachas total sería $R = \sum\limits_{i=1}^{n-1} R_i$. Usando el procedimiento anterior, se llega a que la distribución asintótica nula estandarizada con media $\mu = \frac{2n-1}{3}$ y varianza $\sigma^2=\frac{16n-29}{90}$ es la normal.
	
\subsubsection{Test de bondad del ajuste}
	
	 

	\subsection{Test basados en permutaciones}

\section{Test bayesianos}	
%\end{document}